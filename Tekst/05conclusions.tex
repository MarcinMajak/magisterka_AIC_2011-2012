\section{Summary and conclusions}
\label{cha:Summary}
\subsection{Conclusions from conducted experiments}
In this paper the results of simulation investigations were presented. The main
purpose of five test scenarios was to evaluate prepared classifiers in case of
classification accuracy. Implemented and tested algorithms in this paper are as follows:
\begin{enumerate}
    \item Basic Rough sets algorithm
    \item Rough sets algorithm with modification of decision rules
    \item Genetic Based Fuzzy Logic algorithm
    \item Multistage hybrid classifier using Rough sets and Fuzzy logic
\end{enumerate}
Author intention was to present the whole process of constructing complex
classifier from simpler ones. Researches started with the basic rough sets
algorithm. Because the results of classification were not satisfactory new
algorithm with modification of decision rules was introduced. Significant
improvement was visible for simple problems, but for multidimensional datasets
the classification accuracy remained poor. The next proposal consisted in
constructing multistage hybrid classifier where the power of fuzzy logic and
rough sets reasoning were connected. Original algorithm presented in this
thesis used genetic algorithm for finding the reduct of attributes and the
optimal granulation step $G$ for each attribute in case of rough sets and for
fuzzy logic genetic algorithm determined the best decision rule set. 
Conducted test confirmed that proposed algorithm can compete with other classifiers 
and obtains reliable results. Next paragraphs will shortly summarize each research. 

The main goal of the first simulation was to check how the granulation step $G$
affects the classification accuracy. The efficiency of rough sets algorithm is
determined by the number of \textit{certain} and \textit{possible} rules. Ideally, it would be
great that all rules are \textit{certain} and we have 100\% coverage in the rule set.
Conveyed simulations confirmed the thesis that granulation step has the
greatest impact on the classification results. This parameter must be chosen
very carefully and few factors must be taken into account:
\begin{itemize}
    \item the complexity of the problem, how many attributes are used to
        describe the pattern,
    \item how the granulation intervals are generated, equally for all
        attributes or independently,
\end{itemize}
Another important conclusion from the first test concerns the optimal value for
$G$. It is not worth increasing $G$, good results of classification are
obtained rather for small numbers such as five or six. As the evidence analyze
table \ref{tab:simulation_research_1} where for $G=18$ the classification accuracy
was very poor. Increasing $G$ ensures that the number of \textit{certain} decision rules
is greater, but on the other hand there are more dummy rules (rule with
strength equal to 0). 

Taking into account the fact that the classification accuracy of algorithm
presented in section \label{cha:Simulation_reaearch_1} was not satisfactory,
the recursive rough sets algorithm was proposed to tackle with situation where
for the particular granulation $G$ there is no appropriate rule in the rule sets. From
simulations it can be concluded that this approach improved the classification
significantly and what is more important, thanks to the modification of
decision rules algorithm was not affected by the granulation step $G$. The stability
of classifier is the biggest advantage, but on the other hand the main drawback
is connected with algorithm execution. In situation when the pattern is rejected
from classification algorithm is recursively invoked until a proper rule is
found. This requires much more time than the basic approach without rule
modification.  For simple classification tasks this is not a problem, but for
more complex datasets algorithm computation is much longer.

This thesis concerns the problem of classifying unknown patterns using rough
sets approach and fuzzy logic reasoning. To successfully accomplish this task
new algorithm of fuzzy logic had to be implemented. In the literature there are
many examples of fuzzy logic controllers or classifiers. In this paper genetic
based fuzzy logic classifier was implemented. The main advantages of proposed
solution are as follows:
\begin{itemize}
    \item it manages to construct decision rules without any expert knowledge.
        It means that the input to the system constitutes only raw data from
        dataset without additional information about the number of membership
        functions or their location in the feature space. As the output we
        obtain rule set with the highest classification rate,
    \item it is able to apply attribute reduction,
\end{itemize}
In section \label{cha:Simulation_reaearch_3} it was shown that proposed fuzzy
classifier was able to classify 32 out of 34 testing patterns from iris dataset
with  five rules (presented in fig. \ref{fig:fuzzy_result}).

As fuzzy logic was finished and genetic algorithm turned out to be a good
solution it was decided to connect genetic algorithm with rough sets and check
how this fusion works. The main goal of the heuristic approach was to find an
optimal granulation for each feature independently and try to remove those
attributes that are useless in classification. Conducted tests proved that
this direction of researches is good and gives promising results. Especially it
was visible for high dimensional datasets where the classification rate for
basic rough sets and rough sets with modification of decision rules was very
low. 

The last part in this master thesis consisted in constructing a multistage 
hybrid classifier connecting implemented by the author genetic rough sets
algorithm and genetic based fuzzy logic algorithm. There are many types and
methods of classifier fusion, but in this paper two-phased sequential
classification was proposed. Comparison with different well-known classifiers
gave optimistic results, but of course more profound test are required to fully
prove its usefulness.
\subsection{General conclusions} 
To sum up, this paper presents the result of work in the field of pattern
recognition and classification. It reviewed some of the most representative 
publications connected with rough sets, fuzzy logic and genetic algorithms.
Here an original construction of hybrid classifier has been presented. Attached
results are very promising and encourage for further more complex
investigations. Generally, pattern recognition is not an easy task and many
factors and parameters must be taken into account. Until now, no-one has
managed to invent the classifier with 100\% accuracy. Even if for one dataset
results are satisfactory, but for others classification will be worse.
