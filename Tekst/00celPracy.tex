\section{Introduction}

\subsection{Preface}
\label{cha:Goals}
Genetic algorithms can be computationally expensive as the result of
complexity within particular application or due to the sheer numbers needed to
reach an acceptable solution. In many problems found in the literature the
expense of a genetic algorithm is due to both these factors \cite{bib5},
\cite{bib7}, \cite{bib11}. The genetic algorithm is a global searching process
based on Darwin's principle of natural selection and evolution, so taking into
account this fact, many scientists have noticed that program simulating genetic
algorithms can be implemented in various ways, from
simple sequential program to different method of parallelization. The advent of
multiprocessors has spawned a number of $\mathcal{GA}$ implementations which take advantage
of parallelism available in the algorithm(migration process, subpopulations,
diffusion) \cite{bib8}, \cite{bib9}. In recent years, multi-population genetic algorithms $\mathcal{MGA}$ 
have been recognized as being more effective both in speed and solution quality 
than single-population genetic algorithms $\mathcal{SGA}$. Despite of these advantages,
the behavior and the performance of $\mathcal{MGA}$ is still heavily 
affected by a~judicious choice of parameters, such as connection topology, 
migration method, migration interval, migration rate, population number and many
others. The main problem to reconsider in this topic, is to find the best topology, configuration
parameters and balance between the potential speedup and number of slaves.

\subsection{Main goals of the thesis}
This thesis concerns the problem of parallel genetic algorithms and evolution
programs realisation. The main goal of it is to investigate achievements in this area,
especially paying great importance to the practical usage of presented methods. To
speak more clearly there are two approaches. The first is to make theoretical
analysis of advantages of parallel program realisation and the second is to put
gained knowledge into practice and conduct some experiments to find out the best
parallel genetic algorithm configuration.

As it was said in the previous section, the efficiency of parallel genetic
algorithm in great extend depends on the correctly chosen configuration. In the
literature one can find many examples how only one badly set parameter can
deteriorate algorithm performance; the main problem is with the size of
population because too small results in small diversity of individuals while too
big population wastes computer resources and elongate algorithm execution. Even
the simplest $\mathcal{SGA}$ algorithm has many parameters settings which the
influence on the algorithm result is non-linear. Because this project is the
first step of author's researches in the field of parallel genetic algorithms, it
is justifiable to perform basic experiments on each parameter separately to find
their dependency and check how they influence the results. 

At the end of this section the experiment environment should be characterized in
few words. Given is the problem of optimization; in this thesis it would be used
De Jong's functions(fully described in section \ref{cha:function}) with the
whole knowledge of a~priori global minimum. To find is the global minimum of
each test function with the greatest accuracy to a~priori minimum and in a reasonably
time of computation. To fairly evaluate tested algorithms efficiency
indicators were introduced(more information in section \ref{cha:indicators}).

\subsection{Scope of this project}
This paper comprises of two parts. In the first, the review of
literature and the basic notation used in the whole paper is presented. Sections
\ref{cha:Introduction}, \ref{cha:teoria} are the source of the basic
knowledge about parallel genetic algorithms. There is taxonomy connected with
genetic algorithm introduced and characteristic differences
between $\mathcal{SGA}$ and $\mathcal{PGA}$. Additionally, this part presents
types of parallel genetic algorithm and short overview of parallelization
methods. Sections \ref{cha:PgamathematicTheory},\ref{cha:CoarsedGa} present mathematical background 
of parallel genetic algorithm in case of time realisation, size of population
and number of slave units.


The second part, which is the main point of this thesis, presents experiments analysis.
Paragraph \ref{cha:ExperimentAnalysis} describes the experiment environment
setting and program written in $\mathcal{C++}$ to simulate standard sequential genetic
algorithm and its parallel counterparts. Results with comments are placed in
section \ref{cha:investigation}. Plans for the future work and general
conclusion are placed in sections \ref{cha:FutureWork}, \ref{cha:Summary}
respectively. 

