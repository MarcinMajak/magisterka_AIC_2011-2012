\section{Future work}
\label{cha:FutureWork}
Noways, when multiprocessor and cluster computers are more easily available it
is time to think about possibilities of algorithm parallelization. Of course not every
algorithm is suitable for multiple CPU implementation, but genetic algorithm is
the best in this case. In the literature there can be found many references
about genetic algorithms and its practical parallel implementations
\cite{bib2}, \cite{bib8}, \cite{bib9}, \cite{bib17}. 

It this thesis there were only the basic experiments carried out mainly to
familiarize with the problem and check the advantages of parallel genetic
algorithm realization against simple sequential counterpart. In the future it is strongly recommended to
perform more precise experiment to fully understand behaviour of parallel
algorithm. The main goal of the future work would be a proposal of universal 
algorithm settings, depending on available
resources and problem type. Problems to reconsider:
\begin{itemize}
	\item Use $\mathcal{PGA}$ to solve practical problem. Here, there were used
		test bench functions only to check the accuracy of algorithm solution, but it
		would be great to use it for such problems as: Salesman travelling problem or job scheduling
		problem and then measure possible speedup.
	\item Check the efficiency and accuracy of solution when $\mathcal{PGA}$ is
		implemented as an island model on many real processor units which would require
		creation of LAN network between CPUs. Additional experiment would check
		how the communication between nodes on the LAN affects the final results
	\item Implement library which could be used to automatically deploy task on
		available processors. It would be resemble to well-known solutions such
		as pvm(parallel virtual machine) or mpi(message protocol
		interface)\cite{bib13},\cite{bib23}, but with the ability to run
		load balancing. It would be strongly recommended in cluster
		environment were each node can be occupied by task from different
		problem. 
	\item Different configuration parameters of parallel genetic algorithm. In this
		paper there were only presented results from island type of
		$\mathcal{PGA}$, but as it was mentioned in paragraph \ref{cha:MultiprocessorTaxonomy}
		there are many types depending on available resources and problem
		nature. Future experiments would take into account three configurations
		\begin{enumerate}
			\item Master-slave, when each slaves synchronizes with the master and
				processes evaluation according to master's command
			\item Master-slave, when slaves are only responsible for function
				evaluation. Every generation master sends genomes to its slaves
				and they return fitness evaluation. 
			\item Island model, with various topology.
		\end{enumerate}
	\item Apply some kind of hybridization into implemented program
\end{itemize}

