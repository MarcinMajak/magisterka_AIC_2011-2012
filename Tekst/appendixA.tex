\appendix
\makeatletter
\def\Pref@section{Appendix~}
\def\@seccntformat#1{\csname Pref@#1\endcsname \csname the#1\endcsname\quad}
\makeatother

\section{Program description}
\label{Appendix}
\subsection{Installation requirements}
For the test purposes the simulator in \textit{Python} language was written. To
successfully ran this program few requirements must be met. In this master
thesis this software was executed on Linux platform and here this approach will
be described. Of course it is possible to run the program on Windows, but
proper preparation must be undertaken. Requirement for the software:
\begin{itemize}
    \item \textit{Python} in version at least 2.6
    \item NumPy
    \item SciPy
    \item mlpy library with Gsl. Steps for the proper installation:
        \begin{enumerate}
            \item sudo apt-get install python2.6-dev
            \item go to: \textit{http://www.gnu.org/prep/ftp.html}
            \item click on an ftp link close to your location
            \item find the gsl/ directory and click on it
            \item find the gsl-VERSION.tar.gz file, where version is 1.14 or greater. Click on that file to download it.
            \item In a terminal window extract the tar.gz file using tar -xzf gsl-VERSION.tar.gz and then cd to the ./gsl-VERSION 
                directory
            \item Look at the INSTALL file. It will probably tell you to run ./configure, then make, and then make install
            \item download mlpy from \textit{http://sourceforge.net/projects/mlpy/files/}
            \item unzip file and inside directory run from command line python setup.py install
        \end{enumerate}
\end{itemize}

The whole project is divided into modules and each classifier is implemented as
python class:
\begin{itemize}
    \item BasicClassifiers- this class implements basic classifiers such as:
        LDAC, 3-KNN, MaximumLikelyHood Classifier, Gini Index Classifier, svm
        Classifier
    \item RoughSetsClassifier- this class implements basic rough sets
        classifier. Depending on the chosen module it is an algorithm with
        modification of decision rules or not.
    \item GeneticFuzzyLogicClassifier- this class simulate genetic fuzzy logic
        classifier. In the beginning genetic algorithm is run to obtain the
        best decision rule set and later classification is done
    \item GeneticRoughSetsClassifier- this class implements genetic rough sets
        classifier. It comprises of two parts:
        \begin{itemize}
            \item genetic algorithm for obtaining an optimal partition for each
                feature
            \item classification procedure which uses partition from the
                previous step for pattern recognition
        \end{itemize}
    \item HybridClassifier- this class implements hybrid classifier. This is a
        multistage classifier in which rough sets algorithm is treated as the
        first classifier and fuzzy logic as the second.
\end{itemize}
\subsection{Example usage}
To run each classifier few basic steps must be done. First of all proper
parameters with cross-validation and dataset type must be chosen. Below, a
simple example is presented showing how to run genetic rough sets classifier
fir iris dataset
\lstinputlisting[language=Python]{code/example.py}
\subsection{Results}
In each simulation results of classification for each classifier are saved in
csv file. Names of the files are as follows:
\begin{itemize}
    \item \textit{results/rough\_sets\_classifier.csv} for RoughSetsClassifier
    \item \textit{results/genetic\_fuzzy\_logic\_classifier.csv} for GeneticFuzzyLogicClassifier
    \item \textit{results/genetic\_rough\_sets\_classifier.csv} for GeneticRoughSetsClassifier
    \item \textit{results/hybrid\_classifier.csv} for hybrid classifier
\end{itemize}
After execution of the program each file is created from scratch so be sure
that the previous version of results is save in another location or another
name.

