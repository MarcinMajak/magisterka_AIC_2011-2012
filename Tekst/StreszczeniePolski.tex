\selectlanguage{polish}
\begin{center}
	Streszczenie
\end{center}

Niniejsza praca porusza problem równoległej realizacji algorytmów genetycznych
i~procesów ewolucyjnych. W ostatnich czasach nastąpił znaczący rozwój komputerów
wieloprocesorowych, dzięki czemu pojawiła się możliwość prowadzenia obliczeń
rozproszonych; jako przykład można podać ``cloud computing'' czy też struktury
meshowe. Analizę równoległej realizacji algorytmów można przeprowadzić na dwóch
płaszczyznach: pierwsza to rozkład struktury algorytmu i wydzielenie części, które
mogą zostać zrealizowane równolegle, a także określenie opłacalności tego
zrównoleglenia; druga płaszczyzna to implementacja algorytmu i wybór
odpowiedniego środowiska do testowania. 

Przeprowadzone symulacje skupiają się na zbadaniu efektywności realizacji
równoległego algorytmu genetycznego w stosunku do wersji sekwencyjnej,
szczególnie pod kątem czasu trwania algorytmu, a także uzyskanej dokładności
rozwiązania. Algorytm genetyczny został wzięty do analizy ze względu na
swoją budowę i łatwość zrównoleglenia pewnych operacji, takich jak mutacja,
reprodukcja. Aby przeprowadzić wszystkie symulacje, został napisany program w
języku C++ symulujący prosty sekwencyjny algorytm genetyczny, a także jego
równoległy odpowiednikm, gdzie migracja odbywa się na dwa sposoby: w~pierwszym
przypadku tylko do jednego sąsiada danego procesu slave, natomiast w~drugim do
dwóch sąsiednich procesów slave jednocześnie. Do testowania zostały wykorzystane
dobrze znane w~literaturze przedmiotu funkcje testowe, m.in. funkcje De Jong'a,
ze wględu na swoją różnorodność, a~także występowanie kilku wartości lokalnego
minimum ale tylko jednego globalnego co znacznie utrudnia znalezienie właściwej
wartości przy małej różnorodności osobników. 
Zadaniem stworzonego algorytmu było znalezienie minimum danej funkcji w~określonym przedziale.
Z~racji tego, że znane było globalne minimum pojawiła się możliwość
jednoznacznej oceny wydajności algorytmów za pomocą zaproponowanych wskaźników
efektywności:
\begin{itemize}
	\item Błąd bezwzględny pomiędzy znalezionym rozwiązaniem, a wartością
		a-priori globalnego minimum. Wyróżniono trzy wskaźniki:
		\begin{itemize}
			\item Najmniejsza wartość błędu z~$n$ symulacji 
				\begin{equation}
					B=MIN\left | f^{*}(\underline{x})-f(\underline{x}) \right |
					\label{min1}
				\end{equation}
			\item Największa wartość błędu z~$n$ symulacji 
				\begin{equation}
					W=MAX\left | f^{*}(\underline{x})-f(\underline{x}) \right |
					\label{min3}
				\end{equation}
			\item Średnia wartość błędu z~$n$ symulacji
				\begin{equation}
					A=\frac{1}{n}\sum_{n=1}^n\left | f^{*}(\underline{x})-f(\underline{x}) \right |
					\label{min2}
				\end{equation}
		\end{itemize}
		gdzie $f^{*}(\underline{x})$ oznacza wartość funkcji wyznaczoną przez
		testowany algorytm, a~$f(\underline{x})$ to globalne minimum.
	\item Wariancja błędu z~$n$ symulacji 
		\begin{equation}
			\sigma^2=\frac{1}{n-1}\sum_{i=1}^n\left[A-B\right]^2
			\label{min4}
		\end{equation}
	\item Średni czas trwania algorytmu z~$n$ symulacji, gdzie $T_i$ to $i-ty$ 
		czas symulacji
		\begin{equation}
			T=\frac{1}{n}\sum_{i=1}^nT_i
			\label{min4}
		\end{equation}
\end{itemize}


Symulacje zostały przeprowadzone na platformie Linux z wykorzystaniem
oprogramowania pvm(Parallel Virtual Machine), które pozwala na wirtualizację
kilkunastu procesorów na jednej stacji roboczej pełniącej rolę mastera. W
przypadku algorytmów genetycznych bardzo ważną rolę odgrywają parametry
konfiguracyjne. Źle dobrane mogą wpłynąć znacznie na wydajność, szczególnie na
czas trwania algorytmu, a~także jakość znalezionego rowiązania. Często przed
przystąpieniem do symulacji parametry te ze względu na skomplikowane zależności
wybierane są losowo, jednak w tej pracy konfiguracja została ustalona na
podstawie literatury przedmiotu, analizując dostępne rozwiązania i rezultaty
zaprezentowanych aplikacji. Program stworzony na potrzeby tego projektu był
uruchamiany z wykorzystaniem lini komend, a~do kompilacji zdefiniowano reguły
w~pliku \tescsc{MAKEFILE}. Dokładniejszy opis programu i~wykorzystanych
bibliotek znajduje się w~dołączonym do pracy załączniku. 
Ze względu na stochastyczny charakter działania algorytmów genetycznych każda z
symulacji była powtórzona $200$~razy, a ostateczne wyniki uśredniono. 

Praca została podzielona na dwie części. Rozdziały 2~i~3 stanowią wprowadzenie
do tematyki równoległych algorytmów genetycznych; zostały tam opisane kryteria
podziału, a także sposoby równoległej realizacji wraz z matematycznym
uzasadnieniem opłacalności ich stosowania. Środowisko testowania, a~także
stworzony program zostały przedstawione w rozdziale~4, natomiast właściwe
wyniki badań wraz z komentarzem znajdują się w rozdziale~5. Zostały
wykorzystane tam dwa typy algorytmów genetycznych: pierwszy to prosty
sekwencyjny przypadek, natomiast drugi to implementacja realizująca równoległy
algorytm genetyczny z kilkoma subpopulacjami, określany w literaturze jako
``Gruboziarnisty równoległy algorytm genetyczny''. Symulacje miały na celu zbadanie
ogólnych zależności pomiędzy liczbą CPU, częstością migracji i~liczbą
osobników, a~także ich wpływ na efektywność algorytmu. Propozycje przyszłych
badań oraz ogólne podsumowanie pracy można znaleźć w rozdziałach 6~i~7.

Biorąc pod uwagę fakt, że projekt ten jest pierwszym w dziedzinie równoległej
realizacji algorytmów genetycznych, warto było na początku przeprowadzić
podstawowe badania które miały na celu nakreślić ogólny charakter algorytmów
równoległych w~stosunku do wersji sekwencyjnej, a~także wpływ parametrów
konfiguracyjnych. Przeprowadzone w tym projekcie badania symulacyjne skupiły się
głównie na trzech podstawowych ale bardzo ważnych aspektach realizacji algorytmów równoległych:
\begin{enumerate}
	\item Liczba CPU w stosunku do czasu trwania algorytmu, a~także jakości
		uzyskanego rozwiązania. W przypadku algorytmów genetycznych bardzo
		ważną rolę odgrywa narzut komunikacyjny, który w pewnym momecie może być
		nawet większy niż czas obliczenia funkcji przystosowania danego
		osobnika. 
	\item Częstość migracji- tutaj pojawia się pytanie jaka konfiguracja jest
		optymalna. Czy lepiej jest częściej dokonywać wymiany osobników aby ustrzec
		się przed zbyt szybką konwergencją, jednak z drugiej strony zbyt
		intensywna migracja zwiększa narzut komunikacyjny i~może prowadzić do
		zachwiania stabilności rozwiązania. 
	\item Liczba osobników biorących udział w procesie migracji- podobnie jak w
		poprzednim punkcie należy znaleźć balans pomiędzy narzutem
		komunikacyjnym, a~jakością otrzymanego rozwiązania i~czasem trwania
		algorytmu.
\end{enumerate}
Jak można zauważyć na podstawie powyższych punktów nawet podstawowe parametry
konfiguracyjne mają ogromny wpływ na wydajność algorytmu.

Przeprowadzone w tym projekcie badania potwierdziły, że aby zaimplementować wydajny równoległy
algorytm genetyczny parametry konfiguracyjne muszą być wybrane z dużą rozwagą.
Wiele rzeczy musi być wziętych pod uwagę, takich jak topologia, częstość
migracji czy też liczność populacji. Generalny wniosek płynący z przeprowadzonych 
eksperymentów to opłacalność stosowania $\mathcal{PGA}$, w stosunku do
$\mathcal{SGA}$, jednak należy tutaj podkreślić, że tylko w sytuacji dobrania
odpowiedniej konfiguracji wejściowej. W pracy zostały zaprezentowane przypadki,
w których nawet jeden niewłaściwie dobrany parametr negatywnie wpływał na
wydajność $\mathcal{PGA}$. Zbadany $\mathcal{CPGA}$ był w stanie znaleźć
generalnie o~wiele lepsze, w sensie
jakościowym, rozwiązania dla testowanych problemów optymalizacyjnych niż standardowy
$\mathcal{SGA}$. Prócz tego charakteryzował się ogólnie większą stabilnością, 
a~zatem i większą przewidywalnością i niezawodnością działania. Zjawisko to było
szczególnie widoczne w odniesieniu do problemów o względnie skomplikowanej naturze.

Podsumowując należy zaznaczyć, że niniejsza praca zrealizowała w całości
sformułowane cele, mianowicie: przedstawiona została jednolita systematyka występujących
w literaturze przedmiotu, współbieżnych implementacji $\mathcal{GA}$ wraz z
metodyką doboru konkretnego sposobu realizacji równoległej w zależności od
dostępnych zasobów. Opisana została teoretyczna analiza oczekiwanych ze zrównoleglenia
korzyści wraz z~matemtycznym uzasadnieniem oraz przeprowadzone zostały badania eksperymentalne
wybranej arbitralnie techniki realizacji równoległej, którą w tym przypadku
reprezentował algorytm $\mathcal{CPGA}.


