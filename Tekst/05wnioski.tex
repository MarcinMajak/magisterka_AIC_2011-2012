\section{Summary and conclusions}
\label{cha:Summary}
\subsection{Conclusions from conducted experiments}
In this paper there were four basic simulations carried out focused on 
parameters setting such as population size, migration interval and number of slave units.
They have shown the general relationship between chosen
configuration and algorithm performance. The first simulation indicated
superiority of $\mathcal{PGA}$ over standard $\mathcal{GA}$ in case of algorithm
duration and obtained solution accuracy. The second research confirmed that
there exists the optimal number of slaves. Too many slaves can increase time of
execution without any improvement in solution and spoil diversity inside
populations. The best value for population
counted 300 individuals would be 5 or 6 slaves. One of the most important
experiment in this thesis was to find the best interval migration and how many
individuals should be sent in this process(section \ref{cha:both}). The lower bound
($\mathcal{I}=0$) and upper bound ($\mathcal{I}=1$) has the negative impact on
performance of algorithm especially in respect of time duration and stability of
solution.
The optimal number of individuals to send in migration process for
$\mathcal{PGA}_b$ was from the range of $5$~to~$10$ when migration was triggered after every $20$ 
iterations. Another relatively good solution was obtained for $\mathcal{M}=10$ and
$\mathcal{I}=50$. 

Another issue worth few words of comments is the program implementation which
was used to
simulate ``island model'' of parallel genetic algorithm. To manage communication
between slaves pvm library was applied, one of the simplest and
easiest solution used for dispersed systems, especially on the Linux platform.
Communication logic is implemented on the server side so sending messages
between processes requires only giving process $id$. Another important thing
which was successfully implemented is the synchronization between slaves units.
Without threads and synchronization mechanism it was impossible to ensure that
each slave receives the same number of individuals in migration process, but
with the help of mutex and thread signalization problem was resolved.

\subsection{General conclusions}
k
This paper reviewed some of the most representative publications on parallel
genetic algorithms and showed experiments results to support the thesis that
parallelization is profitable for certain parameters configuration. 
In recent years there have been many experiments carried 
out to show $\mathcal{PGA}$ speedup advantages over simple sequential algorithm. To design
fast and reliable parallel genetic algorithm firstly configuration has to be
chosen carefully. There are many things to reconsider such as topology,
migration rate, number and size of demes. Each parameter affects the quality of
the search and the efficiency of the algorithm in non-linear ways. It seems that
the only way to achieve a greater understanding of parallel $\mathcal{GA}$ is to
study individual parameters independently and check their impact on algorithm
performance.

In particular, the design of parallel $\mathcal{GA}$ involves choices such as using one
population or multiple populations. In both cases, the size of the population
must be determined carefully, and when multiple populations are chosen, one
must also decide how many to use, additionally populations may remain isolated
or may communicate. Communication involves extra costs and extra
decisions on topologies such as how many individuals are exchanged, and the
frequency of communications. The main advantage coming from properly chosen
migration step is improving  diversity in population.

Although, as presented in this paper parallelization of genetic algorithm can be
effectively used to improve performance, some additional experimentation is
necessary in the future to calibrate prepared program to the
particular and hardware environment to fully confirm its usefulness.
