\section{Results of experiments}
\label{cha:Simulation_investugations}
This section presents environment setup and later the results of simulation
investigations. It is important to describe simulation setup so that in the
future someone could repeat test or maybe extend the application.
\subsection{Testing environment}
When it comes to classification task there is always a problem of how to
divide available dataset into training and testing sets. One of the most common 
approach to ensure proper classifier evaluation is cross validation (more
information about cross validation methods can be found in \cite{bib41}). 
There are different types:
\begin{itemize}
    \item \textbf{holdout cross validation}- data set is separated into two sets, called the 
        training set and the testing set. Classifier is trained using the 
        training set only. Then the classifier is asked to predict the output 
        values for the data in the testing set. The advantage of this method is 
        that it is usually preferable to the residual method and takes no longer 
        to compute. However, its evaluation can have a high variance, because
        it depends heavily on which data points end up in the training set and 
        which end up in the test set,
    \item \textbf{Leave-one-out cross validation}- the classifier is trained on all the
        available data except for one point and a prediction is made for that
        point and stored to compute the average error from all points,
    \item \textbf{K-fold cross validation}- the data set is divided into $k$ subsets, 
        and the holdout method is repeated $k$ times. Each time, one of the $k$ subsets 
        is used as the test set and the other $k-1$ subsets are merged together to form 
        a training set,
\end{itemize}
In this thesis $4$-fold cross validation was used (see fig.
\ref{fig:cross_validation}). To ensure that presented
results are reliable each test was repeated 10 times.
\begin{figure}[H]
    \begin{center}
        \includegraphics[width=0.6\textwidth]{fig/cross_validation.jpg}
    \end{center}
    \caption{Example of 4-fold cross validation.}
    \label{fig:cross_validation}
\end{figure}
\label{cha:Simulation_environment}
\subsection{Testing results}
\label{cha:Simulation_results}
The main goal of performed tests was to check classification accuracy for
different algorithms. As stated before three algorithm were used, but in
simulations were carried out. To sum up, the main motivations are as follows:
\begin{itemize}
    \item check the accuracy of classification for basic rough sets algorithm
        and show that in more complex problem it is useless,
    \item show how the modification of decision rules improves the efficiency
        comparing with the basic rough sets algorithm,
    \item check how genetic algorithm can improve the classification accuracy
        for both rough sets and fuzzy logic algorithms,
    \item prove that proposed hybrid classifier can compete with other
        classifiers,
\end{itemize}
\subsubsection{Impact of granulation step $G$ on rough sets algorithm efficiency}
\label{cha:Simulation_reaearch_1}
The aim of this test was to find out how the discretization of feature space
affects the classification accuracy. The number of intervals for each
attribute was the same and denoted as $K_l$, where $l \in (1, \ldots, q)$.
Habernam dataset was taken as the input to the system. Dataset was divided into
training and testing dataset using $4$-fold cross-validation method. The number
of objects to recognize was equal to 153.
Results of the simulation are presented in table
\ref{tab:simulation_research_1} where the notation is as follows:
\begin{itemize}
    \item $G$- granulation step
    \item $O$- total number of correctly recognized objects
    \item $C/CD$- number of patterns for which a \textit{certain} decision rules were
        used/number of correctly recognized patterns using these rules,
    \item $P/PD$- number of patterns for which a \textit{possible} decision rules were
        activated/number of correctly classified objects using these rules,
    \item $V$ number of patterns rejected from classification. There was no
        suitable rule or more rules than one have the same strength but different class
        label.,
    \item $C^*, P^*, V^*$- total number of \textit{certain} , \textit{possible}
        or \textit{dummy} (strength is equal to zero) decision rules, respectively
\end{itemize}
In the experiment, for every feature the initial step of granulation was
changed from 4 to 18 while the factor of its increasing $\epsilona$  was equal 
to one.

Analyzing the results of simulation presented in table \ref{tab:simulation_research_1}
one can see that the quality of the algorithm depends heavily on the step of
granulation $G$ and better results are obtained rather for small $G$.
It can be concluded that increasing $G$ results in growing the number of rules
with the strength equal to zero (parameter $V^*$), because in the learning phase
algorithm is unable to find class representatives. In this case classification is 
impossible and pattern is rejected. The bigger granulation step $G$ is, more
\textit{certain} or \textit{possible} rules are obtained, but on the other hand
the number of cells without any representatives is increasing. Generally, it is
better to keep $V^*$ parameter rather small and correlate its value with $C$
and $P$ factors. Let notice that for $G=18$ the classification accuracy is very
poor even though the number of \textit{certain} rules is the biggest comparing with other
cases. The reason for low classification is connected with $V^*$ where 5794
rules are \textit{dummy}.

\begin{table}[H]
    \caption{Result of simulation for finding the dependency between
    granulation step and classification accuracy}
    \centering
    \begin{tabular}{|c|c|c|c|c|c|c|c|}
        \hline
        $G$ & $O$ & $C/CD$ & $P/PD$ & $V$ & $C^*$ & $P^*$ & $V^*$ \\ \hline \hline
        4&95&0/0&125/95&28&3&7&54 \\ \hline
        5&97&0/0&121/97&32&2&4&119 \\\hline
        6&52&3/1&65/51&85&5&8&203 \\ \hline
        7&74&0/0&91/74&62&3&6&334 \\ \hline
        8&18&2/1&24/17&127&8&9&495 \\ \hline
        9&52&4/4&59/48&90&6&7&716 \\ \hline
        10&57&36/25&38/32&79&12&13&975 \\ \hline
        11&25&5/2&30/23&118&7&9&1315 \\ \hline
        12&45&49/41&9/4&95&19&12&1697 \\ \hline
        13&9&0/0&13/9&140&17&12&2168 \\ \hline
        14&30&9/6&28/24&116&24&14&2706 \\ \hline
        15&29&30/25&17/4&106&27&13&3335 \\ \hline
        16&18&2/1&20/17&131&22&12&4062 \\ \hline
        17&25&3/1&38/24&112&24&17&4872 \\ \hline
        18&9&2/1&11/8&140&25&13&5794 \\ \hline
    \end{tabular}
    \label{tab:simulation_research_1}
\end{table}

\subsubsection{Impact of recursive modification of granulation step $G$ on rough
sets algorithm efficiency}
\label{cha:Simulation_reaearch_2}
In the previous section (\ref{cha:Simulation_reaearch_2}) it was shown that the
granulation step strongly affects the classification accuracy.
Greater $G$ implies that we have more \textit{certain} or \textit{possible}
rules, but on the other the hand number of patterns without rule covering is 
increasing. A lot of patterns are rejected because no proper rule is found. To improve that
situation an algorithm with modification of decision rules is proposed. More
details are presented in section \ref{cha:Algorithm_construction_rough_set_modification}, but generally when an
object is rejected from classification, current $G$ is decreased by $\epsilon=1$ until 
\textit{certain} for \textit{possible} rule is found. The same dataset was used as in
the previous test with the same algorithm settings to show that this approach
is more efficient comparing with the previous algorithm. Results of classification
are presented in fig. \ref{fig:Simulation_research_2} for the basic rough
sets (blue line-BRS) and algorithm with modification of decision rules (orange
line-MRS).
\begin{figure}[H]
    \begin{center}
        \includegraphics[width=\textwidth]{fig/rough_chart.png}
    \end{center}
    \caption{Comparison of Basic Rough sets algorithm with algorithm where
    modification of decision rules is introduced.}
    \label{fig:Simulation_research_2}
\end{figure}

Looking at fig. \ref{fig:Simulation_research_2} one can observe that
the modification of granulation step $G$ during algorithm execution increases 
the classification accuracy and even if algorithm is started with different $G$
the classification stays almost at the same level while in the basic approach the 
greater $G$ implies worse results. Additionally, it is visible that increasing granulation 
step is not the right solution. Even if the number of \textit{certain} or
\textit{possible} rules is greater, the final result is worse. What is more
important, computational time is longer for greater $G$. In this case the optimal 
$G$ would be 5, but for each problem $G$ should be chosen independently because 
it must reflect how patterns are located in the feature space.

The main two disadvantages of proposed rough sets algorithm are as follows:
\begin{itemize}
    \item it uses an arbitrary chosen step of granulation. Modification of 
        decision rule improves classifier quality, but for the prize of
        computational time,
    \item it uses all attributes for creation of decision rules. When the 
        problem is complex then the decision rules are long and tangled. 
        Additionally, some features are useless in classification, instead of valuable
        information they bring noise to the system and deteriorate final
        results,
\end{itemize}

\subsubsection{Impact of number of membership functions on genetic fuzzy logic algorithm efficiency}
\label{cha:Simulation_reaearch_3}
\paragraph{Example of rule generation}
In this section the results of fuzzy logic classifier simulation are presented.
The goal is to show that proposed algorithm construction is correct and gives
satisfactory results. Fuzzy logic algorithm is used in hybrid classifier so it
was required to check its properties and find the best parameter settings,
especially in case of the number of membership functions.

At first, let remind what are the requirements for fuzzy logic classifier in
this thesis. For the input it is provided a dataset without no expert knowledge 
of how to appropriately divide the feature space into fuzzy sets, and how many membership
functions are need. The goal is to find minimal rule set with the possible highest
classification accuracy. 

As the first step, let present how fuzzy logic classifier deals with pattern
recognition and what is the minimal rule set for exemplary Iris dataset. 
At the beginning each feature is divided into 14 membership functions in the same way as 
presented in fig. \ref{fig:fuzzy_example} plus one linguistic variable $DON'T\; USE$. 
The final rule set which was able to classify 32 out of 34 testing patterns is
presented in fig. \ref{fig:fuzzy_result}. From 10 test the following
rule construction occurred 7 times. In this example each attribute
was normalized and after this process feature values were from $<0, 1>$ range. 
\begin{figure}[H] 
    \begin{center}
        \includegraphics[width=\textwidth]{fig/fuzzy_result.png}
    \end{center}
    \caption{Example of rule set generated for Iris dataset}
    \label{fig:fuzzy_result}
\end{figure}
Analyzing figure \ref{fig:fuzzy_result} it is visible that some attributes were
omitted from classification, but the results of classification are quite
satisfactory. Additionally, only five rules were needed for correct pattern
recognition.

\paragraph{Classification accuracy and the number of membership
functions in genetic fuzzy logic algorithm}
The goal of the second part of this test was to check how the number of initial
membership functions affects the final result of classification. There is a
question if it is better to use many small membership functions (for example 14
functions such as presented in fig. \ref{fig:fuzzy_example}}) or only few
functions with greater area coverage. In the first case the solution space is 
much greater than in the second approach so many rules must be created to obtain 
reliable results. Additionally, in most recognition problems we do not need so precise
feature partitions because it can happen that for many created regions we
can not find proper representatives in the training set. 

Parameters for genetic algorithms are the same as presented in table \ref{tab:fuzzy_genetic_parameters}. 
In each simulation the level of partitions $k$ was changed from $7$ to $2$. Few words of explanation should be
written about how $k$ determines the number of membership functions $MF$ for each
attribute. This number is described by eq. (\ref{eq:fuzzy_function_number})
\begin{equation}
    MF = \sum\limits_1^k (n + 1) + 1
    \label{eq:fuzzy_function_number}
\end{equation}
Wine dataset was used as the input to the system. After dividing the set into
three sets(the first one for genetic algorithm, the second and the third for
fuzzy logic training and testing) 45 patterns were used for classification. 

\begin{figure}[H]
    \begin{center}
        \includegraphics[width=\textwidth]{fig/fuzzy_functions.png}
    \end{center}
    \caption{Impact of initial number of membership functions $MF$ per attribute on
    the final fuzzy classification rate}
    \label{fig:fuzzy_functions}
\end{figure}
Looking at the fig. \ref{fig:fuzzy_functions} one can conclude that it is not
worth constantly increasing the number of membership functions ($MF$) per attribute
because for greater $MF$ algorithm obtains worse results. From conducted simulations 
it was noticed that the optimal $k$ value is 4 which leads to $14$
different membership functions per attribute. If not implicitly stated, $k=4$
will be always used in next tests.

\paragraph{Classifiers performance comparison}
Here the genetic fuzzy logic algorithm is compared with rough sets algorithms.
For basic rough sets algorithm granulation step $G$ was set to 4 while for
rough sets algorithm with modification of decision rules this factor was equal
to 7. Parameters for genetic fuzzy logic classifier were the same as presented
in table \ref{tab:fuzzy_genetic_parameters}, parameter $k$ was set to 4. Basic
rough sets algorithm and algorithm with modification of decision rules used all
available features for classification. In case of genetic fuzzy logic
classifier the average number of attributes used in 10 rules was calculated.

The main purpose of this simulation was to check the efficiency of proposed
genetic algorithm, especially how many attributes and how many rules are needed 
for the proper classification. Table \ref{tab:genetic_fuzzy_results} presents
the results of simulation. Notation used in the table is as follows:
\begin{itemize}
    \item $F$- number of attributes in the dataset
    \item $O$- number of objects used for classification
    \item $W$, $B$, $A$, $\sigma$-performance indicators described in section \ref{cha:indicators}
    \item $RSR$- number of objects correctly recognized by basic rough sets algorithm
    \item $RSMRS$- number of patterns correctly recognized by rough sets algorithm
        with modification of decision rules
    \item $GFL$- number of objects correctly recognized by genetic fuzzy logic
        classifier
    \item $FU$- an average number of features used by genetic rough fuzzy logic classifier
        in the best rule set
\end{itemize}
\begin{table}[H]
    \caption{Comparison of accuracy of classification for genetic fuzzy logic classifier with
    rough sets algorithms for different datasets}
    \centering
    \begin{tabular}{|c|c|c|c|c|c|c|c|}
        \hline
        Dataset&$F$&$O$&$W$&$B$&$A$&$\sigma$&$FU$\\ \hline \hline
        \multicolumn{8}{|c|}{RSR}\\ \hline
        haberman&3&77,0&50,0&55,0&53,0&1,9&3,0\\ \hline
        iris&4&39,0&27,0&31,0&29,3&1,5&4,0\\ \hline
        thyroid&5&56,0&41,0&46,0&44,0&1,9&5,0\\ \hline
        bupa&6&87,0&25,0&37,0&30,3&4,3&6,0\\ \hline
        pima&8&192,0&85,0&98,0&91,5&4,9&8,0\\ \hline
        wine&13&45,0&0,0&2,0&0,8&0,8&13,0\\ \hline
        wdbc&30&142,0&10,0&15,0&11,8&2,0&30,0\\ \hline \hline

        \multicolumn{8}{|c|}{RSMR}\\ \hline
        haberman&3&77,0&53,0&58,0&55,5&2,1&3,0\\ \hline
        iris&4&39,0&32,0&38,0&34,8&2,4&4,0\\ \hline
        thyroid&5&56,0&48,0&50,0&49,3&0,8&5,0\\ \hline
        bupa&6&87,0&23,0&35,0&28,8&4,3&6,0\\ \hline
        pima&8&192,0&62,0&69,0&65,5&2,5&8,0\\ \hline
        wine&13&44,5&15,0&23,0&19,3&3,0&13,0\\ \hline
        wdbc&30&142,3&10,0&15,0&11,8&2,0&30,0\\ \hline \hline

        \multicolumn{8}{|c|}{GFL}\\ \hline
        haberman&3&77,0&54,0&58,0&55,9&1,2&8,5\\ \hline
        iris&4&39,0&29,0&39,0&34,3&3,1&8,2\\ \hline
        thyroid&5&56,0&41,0&45,0&43,4&1,2&8,4\\ \hline
        bupa&6&87,0&49,0&58,0&53,4&2,7&8,3\\ \hline
        pima&8&192,0&124,0&140,0&134,1&4,8&7,7\\ \hline
        wine&13&45,0&35,0&42,0&38,0&1,9&8,3\\ \hline
        wdbc&30&142,0&107,0&127,0&119,2&5,6&7,9\\ \hline
    \end{tabular}
    \label{tab:genetic_fuzzy_results}
\end{table}
Results from table \ref{tab:genetic_fuzzy_results} indicate that genetic fuzzy
logic classifier can compete with other classifier. The main advantage is
connected with the number of attributes building decision rules. In each case
some features were removed from the set. Let notice that for more complex
problems, for $d$ greater than $4$ fuzzy logic algorithm obtained the best
results, much better than basic rough sets and rough sets with modification of
decision rules.

\subsubsection{Impact of granulation step $G$ on genetic rough sets algorithm efficiency}
\label{cha:Simulation_reaearch_4}
The goal of this test is to check which approach is better:
\begin{enumerate}
    \item use the same granulation step $G$ for each attribute and additionally
        take all feature into classification,
    \item use different partition for each attribute independently and try to
        remove some features treating them as a noise,
\end{enumerate}
The first approach is simulated by algorithm with modification of decision
rules (see section \ref{cha:Algorithm_construction_rough_set_modification}), 
while the second is genetic rough sets algorithm (described in section
\ref{cha:Multistage}). Results of simulations are placed in table \ref{tab:genetic_rough_results}.
,where the notation is as follows:
\begin{itemize}
    \item $F$- number of attributes in the dataset
    \item $O$- number of objects used for classification
    \item $W$, $B$, $A$, $\sigma$-performance indicators described in section \ref{cha:indicators}
    \item $RSR$ number of objects correctly recognized by basic rough sets
        algorithm
    \item $RSMR$ number of patterns correctly recognized by rough sets
        algorithm with modification of decision rules
    \item $GRR$ number of objects correctly recognized by genetic rough sets
        algorithm
    \item $FU$ number of features used by genetic rough sets algorithm for
        classification.
\end{itemize}

Parameters for genetic rough sets algorithm were the same as presented in table
\ref{tab:rough_genetic_parameters} and for the first rough sets algorithm
granulation step $G$ was equal to $4$ and for algorithm with modification of
decision rules starting granulation value was $7$. These parameters were
selected from the previous simulations because in such configuration the best
results were obtained. For genetic rough sets classifier $K_max$ was set to 7.
\begin{table}[H]
    \caption{Accuracy of classification for genetic rough sets and basic rough
    sets algorithms for different datasets}
    \centering
    \begin{tabular}{|c|c|c|c|c|c|c|c|}
        \hline
        Dataset&$F$&$O$&$W$&$B$&$A$&$\sigma$&$FU$\\ \hline \hline
        \multicolumn{8}{|c|}{RSR}\\ \hline
        haberman&3&77,0&50,0&55,0&53,0&1,9&3,0\\ \hline
        iris&4&39,0&27,0&31,0&29,3&1,5&4,0\\ \hline
        thyroid&5&56,0&41,0&46,0&44,0&1,9&5,0\\ \hline
        bupa&6&87,0&25,0&37,0&30,3&4,3&6,0\\ \hline
        pima&8&192,0&85,0&98,0&91,5&4,9&8,0\\ \hline
        wine&13&45,0&0,0&2,0&0,8&0,8&13,0\\ \hline
        wdbc&30&142,0&10,0&15,0&11,8&2,0&30,0\\ \hline \hline

        \multicolumn{8}{|c|}{RSMR}\\ \hline
        haberman&3&77,0&53,0&58,0&55,5&2,1&3,0\\ \hline
        iris&4&39,0&32,0&38,0&34,8&2,4&4,0\\ \hline
        thyroid&5&56,0&48,0&50,0&49,3&0,8&5,0\\ \hline
        bupa&6&87,0&23,0&35,0&28,8&4,3&6,0\\ \hline
        pima&8&192,0&62,0&69,0&65,5&2,5&8,0\\ \hline
        wine&13&44,5&15,0&23,0&19,3&3,0&13,0\\ \hline
        wdbc&30&142,3&10,0&15,0&11,8&2,0&30,0\\ \hline \hline

        \multicolumn{8}{|c|}{GRR}\\ \hline
        haberman&3&77,0&58,00&61,00&59,50&1,50&2,00\\ \hline
        iris&4&39,0&35,00&39,00&37,00&1,58&2,42\\ \hline
        thyroid&5&56,0&49,00&54,00&51,17&1,91&2,58\\ \hline
        bupa&6&87,0&55,00&61,00&57,67&2,06&2,67\\ \hline
        pima&8&192,0&145,00&155,00&149,42&3,15&3,00\\ \hline
        wine&13&45,0&40,00&44,00&42,67&1,43&2,92\\ \hline
        wdbc&30&142,0&128,00&135,00&132,67&1,65&5,08\\ \hline
    \end{tabular}
    \label{tab:genetic_rough_results}
\end{table}
From table \ref{tab:genetic_rough_results} one can conclude that genetic rough
sets algorithm obtains better results than other algorithms, especially it is
visible for more complex problems such as wine or pima datasets. Additionally,
let analyze the last column $FU$. It determines how many attributes are used in
genetic rough sets algorithm for classification. It is noticeable that some
features are useless and genetic rough sets algorithm is able to find
valuable attributes. Another thing to reconsider is how the complexity of the
problem affects algorithm classification accuracy. Basic rough sets algorithm
tackles quite well with simple problems, for example iris or haberman datasets,
but when the number of attributes is greater than $4$ then the algorithm efficiency
decreases, while genetic rough sets is not affected by this problem and can
deal with complex datasets.

When you compare the performance of genetic rough sets and fuzzy logic
classifier it is visible that genetic rough sets classifier obtains better
results. Taking this into account, hybrid classifier is constructed in such a
way that genetic rough sets algorithm is used as the first stage. Only if
pattern is rejected fuzzy logic classifier is invoked.


\subsubsection{Comparison of hybrid multistage classifier with other classifiers}
\label{cha:Simulation_reaearch_5}
In this section results of simulation for hybrid classifier are presented. 
The accuracy of classification is compared with other classifiers trained and 
tested with the same datasets. The main goal of this simulation was to check if proposed
hybridization (rough set and fuzzy logic) can compete with other classifiers. 
As the source of reference different types of classifiers were chosen to ensure the greatest diversity:
\begin{itemize}
    \item LDAC classifier (Linear Discriminant classifier)- The linear
        discriminant analysis method consists of searching some linear
        combinations of selected variables, which provide the best separation between the
        considered classes. These different combinations are called
        discriminant functions (see example in fig. \ref{fig:ldac_example})
        \begin{figure}[H]
            \begin{center}
                \includegraphics[width=0.5\textwidth, height=0.4\textwidth]{fig/ldac.png}
            \end{center}
            \caption{Example of linear discriminant classifier for 2-dimensional problem}
            \label{fig:ldac_example}
        \end{figure}
    \item 3-KNN Classifier- it is one of the simplest approach in the pattern
        recognition, it classifies objects based on closest training examples
        in the feature space.
    \item Gini index classifier- it is an example of decision tree algorithm
        where the decision is represented in case of decision rules.  Decision
        node specifies a test on a single attribute, leaf node indicates the
        value of the target attribute, arc/edge splits of one attribute and
        indicate the disjunction of test to make the final decision. 
        Decision trees classify instances or examples by starting at the root 
        of the tree and moving through it until a leaf node is reached.
    \item Maximum likelihood classifier- this classifier is commonly used in
        image recognition tasks. It assigns a pixel to a class on the basis of
        its probability of belonging to a class whose mean and covariance are
        modelled as forming a normal distribution in multidimensional feature
        space.
    \item Svm classifier- it is non-probabilistic linear classifier which 
        deals with finding an optimal linear hyperplanes for class separation.
\end{itemize}
Results of simulation are presented in table \ref{tab:final_comparison}, where
notation is as follows:
\begin{itemize}
    \item $O$- number of patterns to be recognized
    \item $W$, $B$, $A$, $\sigma$-performance indicators described in section \ref{cha:indicators}
    \item MACL- Maximum Likelihood Classifier
    \item Hybrid- multistage hybrid rough sets fuzzy logic classifier described
        in section \ref{cha:Multistage_rough_fuzzy}
\end{itemize}
Numbers in bold font indicates this
classifier which obtained the best result for a particular dataset. In cases
when the same result was obtained for more classifier then the number of
attributes is taken into account.

\begin{longtable}{|c|c|c|c|c|c|}
    \caption{Comparison of hybrid rough fuzzy classifier with other common classifiers}
    \label{tab:final_comparison}

    \hline
    \multicolumn{1}{|c|}{Classifier} & \multicolumn{1}{|c|}{$O$} &
    \multicolumn{1}{|c|}{$W$} & \multicolumn{1}{|c|}{$B$} &
    \multicolumn{1}{|c|}{$A$} & \multicolumn{1}{|c|}{$\sigma$} \\ \hline \hline
    \endfirsthead

    \multicolumn{6}{c}%
    {{\bfseries \tablename\ \thetable{} -- continued from previous page}} \\
    \hline 
    \multicolumn{1}{|c|}{Classifier} & \multicolumn{1}{|c|}{$O$} &
    \multicolumn{1}{|c|}{$W$} & \multicolumn{1}{|c|}{$B$} &
    \multicolumn{1}{|c|}{$A$} & \multicolumn{1}{|c|}{$\sigma$}
    \\ \hline
    \endhead

    \hline \hline \multicolumn{6}{|r|}{{Continued on next page}} \\ \hline
    \endfoot

    \endlastfoot
        \multicolumn{6}{|c|}{iris} \\ \hline
        LDAC&39,0&36,0&38,0&36,8&0,8\\ \hline
        KNN&39,0&34,0&38,0&36,2&1,8\\ \hline
        GINI&39,0&31,0&38,0&35,5&2,9\\ \hline
        MACL&39,0&34,0&38,0&36,3&1,5\\ \hline
        SVM&39,0&24,0&26,0&25,0&1,0\\ \hline
        Hybrid&39,0&35,0&39,0&\textbf{37,0}&1,6\\ \hline \hline

        \multicolumn{6}{|c|}{Bupa} \\ \hline
        LDAC&87,0&56,0&62,0&59,3&2,8\\ \hline
        KNN&87,0&55,0&65,0&59,3&4,0\\ \hline
        GINI&87,0&54,0&56,0&55,0&0,7\\ \hline
        MACL&87,0&48,0&56,0&52,5&3,2\\ \hline
        SVM&87,0&53,0&63,0&\textbf{59,5}&3,8\\ \hline
        Hybrid&87,0&55,0&62,0&58,8&2,6\\ \hline \hline

        \multicolumn{6}{|c|}{Pima} \\ \hline
        LDAC&192,0&146,0&149,0&147,5&1,5\\ \hline
        KNN&192,0&131,0&136,0&132,3&2,2\\ \hline
        GINI&192,0&123,0&136,0&130,5&5,3\\ \hline
        MACL&192,0&142,0&145,0&143,3&1,1\\ \hline
        SVM&192,0&128,0&141,0&134,8&4,7\\ \hline
        Hybrid&192,0&146,0&152,0&\textbf{149,0}&2,1\\ \hline \hline

        \multicolumn{6}{|c|}{haberman} \\ \hline
        LDAC&77,0&57,0&58,0&57,5&0,5\\ \hline
        KNN&77,0&54,0&57,0&55,8&1,1\\ \hline
        GINI&77,0&45,0&58,0&52,3&4,7\\ \hline
        MACL&77,0&57,0&58,0&57,8&0,4\\ \hline
        SVM&77,0&55,0&58,0&56,5&1,1\\ \hline
        Hybrid&77,0&58,0&63,0&\textbf{60,0}&2,1\\ \hline \hline

        \multicolumn{6}{|c|}{wdbc} \\ \hline
        LDAC&142,0&133,0&138,0&135,5&1,8\\ \hline
        KNN&142,0&127,0&134,0&131,3&2,7\\ \hline
        GINI&142,0&128,0&131,0&128,8&1,3\\ \hline
        MACL&142,0&133,0&139,0&\textbf{135,8}&2,2\\ \hline
        SVM&142,0&127,0&135,0&131,8&2,9\\ \hline
        Hybrid&142,0&127,0&138,0&134,3&4,3\\ \hline \hline

        \multicolumn{6}{|c|}{thyroid} \\ \hline
        LDAC&56,0&47,0&51,0&49,3&1,8\\ \hline
        KNN&56,0&48,0&52,0&50,8&1,6\\ \hline
        GINI&56,0&48,0&53,0&50,5&1,8\\ \hline
        MACL&56,0&50,0&54,0&51,0&2,1\\ \hline
        SVM&56,0&50,0&53,0&51,0&1,2\\ \hline
        Hybrid&56,0&50,0&55,0&\textbf{52,0}&2,1\\ \hline \hline

        \multicolumn{6}{|c|}{wine} \\ \hline
        LDAC&45,0&41,0&43,0&43,0&0,7\\ \hline
        KNN&45,0&27,0&35,0&39,9&3,0\\ \hline
        GINI&45,0&37,0&41,0&39,0&2,0\\ \hline
        MACL&45,0&43,0&45,0&\textbf{44,3}&0,8\\ \hline
        SVM&45,0&38,0&44,0&41,3&2,4\\ \hline
        Hybrid&45,0&41,0&43,0&43,0&0,7\\ \hline
\end{longtable}

From table \ref{tab:final_comparison} one can conclude that hybrid classifier
obtains quite good results comparing with other classifiers. From seven
datasets four times it was the best. In other cases results are comparable.
What is more important hybrid classifier is able to classify pattern with
reduced number of attributes. In this case created decision rules are simpler
and more readable for user. This is especially important in medicine where
physician is provided with decision rules and basing on them makes the final
diagnosis. Another thing to reconsider is the stability of proposed classifier.
It uses genetic algorithm for rule construction, so taking into account its
random nature hybrid classifier can be unstable, but simulations show that this
is not a problem. Appropriate number of generations for genetic algorithm
assures the proper convergence and as the consequence hybrid classifier is stable. 

To prove that proposed hybridization is effective let compare results from
tables \ref{tab:final_comparison}, \ref{tab:genetic_rough_results}. In every
case the results of classification were better than for single genetic rough
sets algorithm. It means that if pattern was rejected from classification in
the first stage in some cases it was possible to classify pattern by fuzzy
logic classifier.

