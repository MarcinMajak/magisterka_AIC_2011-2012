\section{Future work}
\label{cha:FutureWork}
In this thesis the basic parameters and settings were checked for rough sets
algorithm and fuzzy logic algorithm. In the future, it is strongly recommended 
to carry out more profound simulations to fully understand behaviour of
algorithm in different environment and settings. In this thesis to evaluate
classifiers well-known datasets from \textit{UCI} Repository. This offers
reliable environment for testing, but the next step in the future work is to
apply proposed algorithm into real life problem, such as optimal control or
image pattern recognition. Using rough sets properties it would be advisable to
detect tumor tissue on CT or MRI images or bone structures for further 3D
reconstruction.

Another thing to reconsider in the next researches is how to generate partition
of feature. Here, genetic algorithm was used to find the reduct of attributes
and the number of intervals for each attribute independently. Results of
simulations confirmed the usefulness of this approach, but here arises the
question if there is another solution for finding an optimal feature
granulation. Two possible future tests:
\begin{enumerate}
    \item Testing the classification accuracy of rough sets algorithm when
        granulation is based on the frequency of patterns in the training
        dataset. This approach assumes that for clusters with many patterns the
        granulation will be more precise while in other places it would be
        sparse.
    \item Testing the classification accuracy of rough sets algorithm when
        granulation is determined by fuzzy logic and triangular membership
        functions. A concept of fuzzy discretization of feature space for a
        rough sets theoretic classifier is presented in \cite{bib100}
\end{enumerate}

The last, but not least aspect of the future work is to check different types
of classifier hybridization. In this thesis rough sets algorithm was the most
important and only in cases when pattern was rejected fuzzy logic classifier
was used. It would be required to simulate different scenarios of classifier
ensemble, for example majority voting with fuzzy logic, rough sets and neural
network or 3-KNN classifiers. Additionally, it would be great to compare
different algorithms for feature reduction with genetic approach used in this
paper. 
