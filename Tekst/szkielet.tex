\documentclass[12pt,a4paper]{article}
\addtolength{\textwidth}{1.5cm} 
\addtolength{\textheight}{1cm} 
\usepackage[T1,math]{english}
\usepackage{palatino}				% czcionka palatino
\usepackage[utf8]{inputenc}
\usepackage[pdftex]{graphicx}
\usepackage{amsmath}					% matematyka
\usepackage{url}
\usepackage[toc,page]{appendix}
\usepackage{slashbox}				% do podzielonej na pol komorki w~tabeli
\usepackage{fancyhdr}				% do nagłówkow, linia w~nagłówku
\usepackage{longtable}				% duze tabele > 1 strona
\usepackage[splitrule]{footmisc}	% ustawienia stopki
\usepackage{rotating}				% do obracania obiektów
\usepackage{indentfirst}	% na nazwie sekcji pierwszy akapit ze wcieciem
\usepackage{float}
%\usepackage{polski}
\usepackage[T1]{fontenc} %dziwna szerokosc czcionki
\usepackage[english]{babel}
\usepackage[polish]{babel}

\usepackage[
breaklinks=true,
colorlinks=true,
citecolor=black,
linkcolor=black,
anchorcolor=black,
urlcolor=black
]{hyperref}						% linki w~spisie tresci, rysukacj etc.
\usepackage[all]{hypcap}	% czasami hyperref Ÿle podaje \ref

\usepackage{textcomp} %wymagane przez listings - upquote=true

\usepackage{listings}
\lstset{
    language=Java,
    basicstyle=\scriptsize,
    aboveskip={1.5\baselineskip},
    columns=fixed,
    showstringspaces=false,
    extendedchars=true,
    breaklines=true,
    tabsize=4,
    prebreak = \raisebox{0ex}[0ex][0ex]{\ensuremath{\hookleftarrow}},
    showtabs=false,
    showspaces=false,
    showstringspaces=false,
    identifierstyle=\ttfamily,
    keywordstyle=\color[rgb]{0,0,0},
    commentstyle=\color[rgb]{0,0,0},
    stringstyle=\color[rgb]{0,0,0},
    numbers=left,
    numberstyle=\tiny,
    stepnumber=1,
    numbersep=5pt,
    captionpos=b,
}

\renewcommand*\thelstnumber{\the\value{lstnumber}:} % dwukropek po numerze linii

%\usepackage[small,bf]{caption} % mniejsza czcionka dla podpisów

\sloppy		% mniej staranne łamanie wierszy, zapobiega wystawaniu wyrazom po za 
% szaltę dokumentu

\headheight=15pt
\pagestyle{fancy}
\fancyhead[R]{\small \thepage}
\fancyhead[C]{\small \leftmark}
\fancyhead[L]{}
\fancyfoot{}
\renewcommand{\sectionmark}[1]{%
\markboth{\thesection.\ #1}{}}
\renewcommand{\headrulewidth}{0.4pt}
\renewcommand{\footrulewidth}{0.4pt}


\renewcommand{\textfraction}{0.05}
\renewcommand{\baselinestretch}{1.05}

\newcommand{\ninter}{\renewcommand{\baselinestretch}{1.5}}
\newcommand{\sinter}{\renewcommand{\baselinestretch}{1.0}}
\newcommand{\eng}[1]{(ang.\ \emph{#1})}
\renewcommand\labelitemi{\textbullet}
\renewcommand\labelitemii{$\circ$}  
\renewcommand\labelitemiii{---} %{\textasteriskcentered}
\renewcommand\labelitemiv{---}  %{\textperiodcentered} 

\renewcommand\figurename{Fig.}
\renewcommand\tablename{Table}
\renewcommand\lstlistingname{Listing}
\renewcommand\lstlistlistingname{Code listing}

\makeatletter
\renewcommand{\fnum@table}{\textbf{\tablename~\thetable}}
\makeatother

% numerowaie obrazkow 1.1 1.2 1.3... 2.1 2.2 2.3
\renewcommand{\thefigure}{\arabic{section}.\arabic{figure}}
\renewcommand{\theequation}{\arabic{section}.\arabic{equation}}
\let\stdsection\section
\renewcommand\section{\setcounter{figure}{0}\setcounter{equation}{0}\setcounter{table}{0}\setcounter{lstlisting}{0}\stdsection}

%numerowanie tabel
\renewcommand{\thetable}{\arabic{section}.\arabic{table}}


%%%%%%%COUNTER DO TABEL%%%%%%%%%
%\newcounter{tableindex}
%\newcommand{\tindex} { \thetableindex \stepcounter{tableindex} }
%DOCUMENT%%%%%%%%%%%%%%%%%%%%%%%%%%%%%%%%%%%%%%%%%%%%%%%%%%%%%%%%%% 

%%%%%%%%TO~NALEZY UZUPELNIC%%%%%%%%%%%%%%%%
\def\temat{Efficiency evaluation of parallel genetic algorithms and
evolutionary programs}
\def\tematen{Badanie efektywności równoległej realizacji algorytmów genetycznych
i programów ewolucyjnych}
\def\autor{Marcin Majak}
\def\prowadzacy{Andrzej Żołnierek, PhD, W4/K2}
\def\opiekun{dr inż. Andrzej Żołnierek, W4/K2}
\def\rok{2010}

\newcommand{\mojaSugestia}[1]{\textcolor{blue}{#1}}
\newcommand{\mojaZmiana}[1]{\textcolor{green}{#1}}
\newcommand{\kodWTexcie}[1]{\texttt{#1}}

\usepackage{placeins}  % to many unprocessed floats - problem z numerowaniem wiecej niz 20 obrazkow

\begin{document}

%numerowanie listingów
\renewcommand{\thelstlisting}{\arabic{section}.\arabic{lstlisting}}

%po wpisaniu makr przed \begin{document} strona
%tytulowa uzupelni sie automatycznie

%Mon Dec 18 16:18:58 CET 2006
%autor: Krystian Piećko
\titlepage
%\addtolength{\hoffset}{-1cm}
\addtolength{\textheight}{1.8cm}
%\addtolength{\footskip}{-2cm}
\addtolength{\textwidth}{0.2cm}
%\addtolength{\oddsidemargin}{-1.2cm}
\addtolength{\topmargin}{-2.2cm}
\enlargethispage{3cm}
\begin{center}
	\begin{Huge}
		\textsc{Politechnika Wrocławska}
	\end{Huge}

	\begin{huge}
		\vspace{1ex}
		\textsc{Wydział Elektroniki}
	\end{huge}
	\rule[-0.3ex]{\textwidth}{1pt}
	
	\begin{flushleft}
		\begin{large}
			\vspace{0.4cm}
			\textsc{Kierunek: Informatyka (ang.)} \newline
		\end{large}
	\end{flushleft}

	\begin{center}
		\begin{Huge}
			\vspace{1.7ex}
			\textsc{\textbf{Praca magisterska}} 

		\end{Huge}
	\end{center}
	\vspace{1cm}
		
	%\vspace{1.5cm}
	\begin{flushright}
		\begin{minipage}[t][4cm][t]{11cm}
			\begin{center}
				\begin{large}
					\tematen
				\end{large}
			\end{center}
			%\newline
			\vspace{0.3cm}
			\begin{center}
				\begin{large}
					\temat
				\end{large}
			\end{center}
		\end{minipage}
	\end{flushright}
				
	\begin{flushright}
		\begin{minipage}[t]{10cm}
			\begin{flushleft}
				\begin{large}
					%\vspace{2ex}
					\vspace{0.3cm}
					\begin{large}
						\textsc{Autor:}\newline
						\autor\newline
					\end{large}
					
					\vspace{1.5cm}
					\textsc{Opiekun:}\newline
					\opiekun\newline
					
					\vspace{0.5cm}
					\textsc{Ocena pracy:}\newline
				\end{large}
			\end{flushleft}
		\end{minipage}
	\end{flushright}
	\vfill
	\rule[-0.3ex]{\textwidth}{1pt}
	WROCŁAW \rok
\end{center}
\newpage
\clearpage
\enlargethispage{-3cm}
\addtolength{\textheight}{-1.8cm}
\addtolength{\textwidth}{-0.2cm}
\addtolength{\topmargin}{2.2cm}


\newpage
\selectlanguage{polish}
\begin{center}
	Streszczenie
\end{center}

\newpage

%Mon Dec 18 16:18:58 CET 2006
%autor: Krystian Piećko
\selectlanguage{english}
\titlepage
%\addtolength{\hoffset}{-1cm}
\addtolength{\textheight}{1.8cm}
%\addtolength{\footskip}{-2cm}
\addtolength{\textwidth}{0.2cm}
%\addtolength{\oddsidemargin}{-1.2cm}
\addtolength{\topmargin}{-2.2cm}
\enlargethispage{3cm}
\begin{center}
	\begin{Huge}
		\textsc{Wroclaw University of Technology}
	\end{Huge}

	\begin{huge}
		\vspace{1ex}
		\textsc{Faculty of Electronics}
	\end{huge}
	\rule[-0.3ex]{\textwidth}{1pt}
	
	\begin{flushleft}
		\begin{large}
			\vspace{0.4cm}
			\textsc{AREA: Teleinformatic(TIN)} \newline
		\end{large}
	\end{flushleft}

	\begin{center}
		\begin{Huge}
			\vspace{1.7ex}
			\textsc{\textbf{Engineering project}} 

		\end{Huge}
	\end{center}
	\vspace{1cm}
		
	%\vspace{1.5cm}
	\begin{flushright}
		\begin{minipage}[t][4cm][t]{11cm}
			\begin{center}
				\begin{large}
					\temat
				\end{large}
			\end{center}
			\vspace{0.3cm}
			\begin{center}
				\begin{large}
					\tematen
				\end{large}
			\end{center}
		\end{minipage}
	\end{flushright}
				
	\begin{flushright}
		\begin{minipage}[t]{10cm}
			\begin{flushleft}
				\begin{large}
					%\vspace{2ex}
					\vspace{0.3cm}
					\begin{large}
						\textsc{Author:}\newline
						\autor\newline
					\end{large}
					
					\vspace{1.5cm}
					\textsc{Supervisor:}\newline
					\prowadzacy\newline
					
					\vspace{0.5cm}
					\textsc{Grade:}\newline
				\end{large}
			\end{flushleft}
		\end{minipage}
	\end{flushright}
	\vfill
	\rule[-0.3ex]{\textwidth}{1pt}
	WROCŁAW \rok
\end{center}
\newpage
\clearpage
\enlargethispage{-3cm}
\addtolength{\textheight}{-1.8cm}
\addtolength{\textwidth}{-0.2cm}
\addtolength{\topmargin}{2.2cm}


\newpage
\thispagestyle{empty}
\begin{center}
	Abstract
\end{center}

This paper concerns the problem of parallel genetics
algorithms and evolutionary programs realization. In the last few decades multiprocessor
computers have spawned development of dispere computation processes, to name
only the few it is cloud computing or mesh grid computers. Design of parallel
algorithms can be carried out in two steps. The first is algorithm's structure
analysis and distinction of processes which can be realised in parallel, with
additional consideration of parallelisation profitability. The second step involves algorithm's
implementation and testing environment setup.

Simulations are focused on efficieny analysis between
parallel genetic algorithm and its sequential counterpart, especially in case
of algorithm's duration and solution accuracy. Genetic algorithm was used
because of its structure and its simplicity to realised some methods in parallel
such as: mutation, reproduction, etc. To manage all the simulations program was
written in C++ simulating sequential and parallel genetic algorithms. For
testing, well known in the literature test functions were used mainly De Jong's
functions and in order to fairly evaluate results efficiency indicators were
proposed in this paper. 

All test were performed on Linux platform with the help of pvm(Parallel Virtual
Machine) framework, which allows virtualization of few processors on one machine
called master. Taking into account the stochastic nature of genetic algorithm
each simulation was repeated $200$ times and the final results represent the
average value. 

This paper is divided into two parts. Sections 2~and~3 are the introduction into
parallel genetic algorithm subject matter. There are listed types of parallel algorithm
and method of parallelisation with mathematical justification. Testing
environment setup and program used in simulations are described in section~4,
while the results of experiments with comments are placed in section~5. 
There were used two types of algorithms: the first is simple sequential while 
the second is parallel genetic algorithm with few subpopulations. The main 
goal of simulations was to find the relationship between number of slave units, 
the interval of migration and number of individuals and their impact on algorithm
efficiency. Proposal of future work as the continuation of this topic and the
final summary can be found in sections 6~and~7.


\newpage
\tableofcontents
\newpage

%od tego miejsca mozna zaczac wpisywac tresc.
%polecam korzystac z~polecenia input i~kazdy rozdzial robic w~innym pliku
%tak jak jest wciagnieta titlepage

\newpage
\section{Introduction}
\subsection{Preface}
\label{cha:Goals}
With the improvement of computers we are able to tackle with high dimensional 
problems in control or pattern recognition task. At first glance everything
looks so easy, but when we go deeper into a problem more and more problems 
become evident and unavoidable. For example pattern can be described by many 
attributes. Some of them are very meaningful while others brings only noise 
and distortion. This is the role of the algorithm to pick valuable attributes,
but finding a minimal attribute reduct is $\mathcal{NP}$-hard problem. Another
obstacle is connected with overlaps in the attribute space. When attributes 
are easily separable even a simple classifier works perfectly, but for more 
tricky cases very sophisticated approaches has to be applied.

For the past years many scientist in the world tried to invent new algorithms 
to improve the process of classification or problem of optimal control. Ones 
of the common solutions found in the literature are neural networks, fuzzy
logic or evolutionary algorithms such as genetic algorithm. Because simple 
approaches failed in more complicated problem, scientist tried to applied 
algorithms for dimensionality reduction and merge abilities of single
classifier into combined one. This improved the quality of classification 
significantly, but until then no one has managed to invent such a classifier 
that will make no mistakes.

This thesis touches the broad topic of pattern recognition task which is 
very difficult and demanding problem in every aspect of science. Generally, 
pattern classification is about assigning label to an unknown object based 
on the available knowledge. It can be compared to the capability of human 
brain which is able to put certain scenario into context and identify 
distinguishable object components. The whole process of classification can 
be broken down into few parts and each phase has a significant impact on the
final results (more detailed description will be presented in \ref{cha:pattern
recognition}.

\subsection{Main goals of the thesis}
This thesis concerns the problem of pattern recognition. As it was previously
said in the literature one can find many algorithms used for object
classification, while here the rough sets and fuzzy logic algorithms are used 
and investigated. General purpose of this thesis is to propose hybrid
classifier using the power of fuzzy logic and rough sets. First of all the
mathematical background of these algorithms will be presented and later
simulation investigation will be carried out to prove the usefulness of
proposed fusion. It should be noted that for the simulation purposes author
implemented basic fuzzy logic, rough sets algorithms and created a hybrid
classifier.

At the end of this section, the experiment environment should be characterized in
few words. Given is the problem of pattern recognition: in this thesis it would
be used datasets from $UCI$ Repository (described in section
\ref{cha:experiment}. Those datasets are broadly available and everyone in the
future would be able to repeat the simulation investigation and compare the
results. To find is the algorithm classification accuracy on the given dataset
and comparison with other approaches found in the literature. 

\subsection{Scope of this project}
This paper comprises of two parts. In the first, the review of
literature and the basic notation used in the whole paper is presented. Sections
\ref{cha:Introduction} is the source of the basic
knowledge about pattern recognition. Sections \ref{cha:Rough_set}, \ref{cha:Fuzzy_logic}
presents the description of algorithms used in this thesis. Additionally, this
part shows the algorithm construction steps using pseudo-code.

The second part, which is the main point of this thesis, presents experiments analysis.
Paragraph \ref{cha:ExperimentAnalysis} describes the experiment environment
setting and program written in $\mathcal{C++}$ to simulate standard sequential genetic
algorithm and its parallel counterparts. Results with comments are placed in
section \ref{cha:investigation}. Plans for the future work and general
conclusion are placed in sections \ref{cha:FutureWork}, \ref{cha:Summary}
respectively. 


\section{Heuristic methods of optimization }
\label{cha:Introduction}
\subsection{Heuristic algorithms}
\label{cha:HeurysticAlgorithms}
In the last decade there has been a great increase of interest in heuristic
method of optimization to solve hard problems in the case of complexity and time computation.
To name only the few of well known methods it is $\mathcal{SA}$(Simulated
Annealing), Ant algorithms or $\mathcal{TS}$ (Tabu Search). Algorithms presented
above, have many practical adaptations which are broadly described in the literature, 
but in this paper $\mathcal{GA}$ (Genetic algorithm) is taken into closer look
in respect of obtained solution accuracy, time of evaluation and possible parallelization advantages.

\subsection{Genetic algorithm}
\label{cha:GeneticAlgorithm}
Genetic Algorithm is an element of evolutionary computation, which is a~rapidly 
growing area of soft computing. $\mathcal{GA}$ is based on the principles of natural selection 
and genetic modification. As optimization methods, $\mathcal{GA}$ operates on a population of points, 
designated as individuals. Each individual of the population represents a~possible 
solution of the optimization problem. Individuals are evaluated depending upon 
their fitness which indicates how well an individual of the population 
solves the optimization problem. To sum up, $\mathcal{GA}$ has the following general 
features:
\begin{enumerate}
\item $\mathcal{GA}$ operates with a population of possible solutions (individuals) 
instead of a single individual. Thus, the searching process can be carried out in a
parallel form or sequentially. 
\item $\mathcal{GA}$ is able to find the optimal or sub-optimal solutions in complex and large search 
spaces. Moreover, it can be applied to nonlinear optimization problems with 
constraints defined in discrete or continuous search spaces. 
\item $\mathcal{GA}$ examines many possible solutions at the same time, so there is a~higher probability
that the search process can converge to an optimal solution. 
\end{enumerate}

   There are four main parts in each $\mathcal{GA}$ process to reconsider: the problem representation or encoding, 
fitness or objective function definition, fitness-based selection, and evolutionary reproduction of 
candidate solutions (individuals or chromosomes). To successfully implement
genetic algorithm in programming language one has to define an encoding method,
fitness function, selection method, and reproduction method as well as criteria rules 
for the $\mathcal{GA}$ formulation. 


Genetic algorithms are widely used as a search techniques in the various fields \cite{bib6}, \cite{bib14},
\cite{bib16}. The success of a~genetic
 algorithm can be quantified by estimating the cost, time required 
 and the quality of final obtained solution. In the literature there can 
 be found many examples of how $\mathcal{GA}$ is useful in solving hard optimizations problems, 
 but beside unquestionable advantages there also exist downsides. A traditional
 $\mathcal{GA}$ without any diversity maintenance mechanism often suffers from
 getting stuck on the suboptimal peaks, because 
 almost the entire $\mathcal{GA}$ population would have converged to a single peak, as a result 
 of the rapid loss of population diversity. To verify $\mathcal{GA}$
 implementations quality, programs are very often tested with functions which
 have many local minimum, but only one global minimum \cite{bib25}.
 
 Implementation of a successful genetic algorithm requires optimal parameter setting
 which in many situations turns out to be very complicated because of great
 number of parameters and their dependencies. For example, for small population
 the genetic algorithm tends to make large mistakes(it has small diversity of
 individuals so very often it can stuck in one of the local minimum), while for the grater size
 of population it is capable of~discriminating bad solutions, but for the prize
 of algorithm duration and resources usage. There is a great deal of~work showing how to set
 the optimal parameters in an evolutionary algorithm to obtain required speedup
 and solution accuracy \cite{bib1}.

\subsection{Parallel Genetic algorithms}
\label{cha:ParallelGeneticAlgorithm}
Multi-agent systems have attracted a considerable amount of interest in recent
years \cite{bib17}, \cite{bib21}, \cite{bib22}. For many real world problems, 
these applications showed appealing properties such as robustness, 
increasing capabilities and failure tolerance. Additional reason which spawned
multi-agent systems development is that very large amount of data or data arising at 
different geographical locations are difficult to be handled by traditional, 
sequential and centralized systems. Parallel and distributed architectures have
become inevitable for the reason of computation abilities. Some 
studies have been done on parallel and distributed learning before, but nowadays
when more advanced computer are created, distributed computing has become very
popular. Now, many paradigms like fuzzy logic, neurocomputing, 
genetic algorithm, as elements of soft computing, can augment stochastic
learning procedures to create autonomous control system \cite{bib2}, \cite{bib4}

In case of $\mathcal{GA}$, parallelization can be achieved by creating number of 
separate populations which exchange genetic information during migration
process. This $\mathcal{PGA}$(Parallel Genetic Algorithm) variant is known as migration 
model. The alternative master-slave model uses only one population on master processor
which assigns some computations (usually the evaluation of individual) to
slaves, so parallelization affects only evaluation of the population. Each slave processor
receives an individual to evaluate while master is responsible for other parts of $\mathcal{GA}$ process
such as: mutation, crossover, replacement and exchanging information between slaves
which involves collecting results and sending additional individuals to slaves. This method is useful 
especially for large-scale problems and for computationally demanding evaluation operators. 
Its main drawback is the idleness of slave processors during other phases of
genetic algorithm and high communication ratio between master and slave units. 

The second solution of $\mathcal{PGA}$ which is called ``island model'' can be
described as each processor maintains its own
population of individuals and evaluates it using the same genetic
operators used in simple sequential genetic algorithm. Occasionally migration occurs - every processor 
sends a part of its population (”best” individuals) to all 
other units and receives their representatives as well. On 
every island ”worst” individuals are replaced by newcomers.
The presented approach is also called ”pure” migration 
model and its type is strongly determined by how the migration process is taking
place. It is possible to choose different migration schemes 
i.e. sending of individuals only to neighbouring islands, 
migrating random groups of representatives or replacing random individuals. The
main advantage of this solution is the low communication ratio between master and
slaves because each processor evolves its own population and does not waste time on
waiting for master command as it was described in model of master-slave with one
population.

The main goal of this paragraph was to present the
basic types of parallel genetic algorithms, but this short description is
not enough so in section \ref{cha:PgaTypes} this
topic will be tackled in greater details, paying attention to parallel genetic time
realisation and possible speedup. 

\section{Algorithm construction}
\label{cha:Algorithm_construction}
In this section algorithm construction will be presented. In this thesis four
algorithms were used for pattern recognition task. Here we start with the basic
algorithm of rough sets, later present rough sets algorithm with modification
step and at last multistage hybrid algorithm consisting of genetic algorithm,
rough sets and fuzzy logic will be shown in greater details. 
\subsection{Rough sets algorithm construction}
\label{cha:Algorithm_construction_rough_set}
The basic rough sets algorithm with constant step of granulation
can be summarized in six steps:
\begin{enumerate}
    \item If the attributes are the real numbers then the granulation preprocessing 
        is needed first. After this step, the value of each attribute is represented 
        by the number of interval in which this attribute is included. For each
        attribute from $l=(1, \ldots , q)$ we choose the same numbers of
        intervals $K_l$ called step of granulation $G$. For the $l$-th attribute 
        denoted by $v^l_{p_l}$ define
        its $p_l$ interval from $p_l=(1, \ldots, K_l)$
    \item Using training dataset construct set $FOR(C)$ of decision rules of
        the following form:
        $$IF \, (x^1 = v_{p_1}^1) \, AND \, (x^2=v_{p_2}^2) \, AND \, \ldots \,
        (x^q=v_{p_q}^q) \, THEN \, \Psi(S, x)=j$$
        Each generated rule is evaluated and strength factor is assigned
        accuracy of approximation (see section \label{cha:Rough_sets_indicators})
    \item For the created set of formulas $FOR(C)$ for each $j=1, \ldots, m$ we
        calculate lower, upper approximation and the boundary region.
    \item In order to classify pattern $x$ we look for matching rules in the
        set $FOR(C)$ (the left condition is fulfilled by its attributes).
    \item If there is only one matching rule, then we classify this pattern $x$ 
        to the class  which is indicated by its decision attribute $j$, 
        because for sure such rule is belonging to the lower approximation of all rules 
        indicating $j$, this rule is certain.
    \item If there is more then one matching rule in the set $For(C)$, 
        it means that the recognized pattern should be classified by the 
        rules from the boundary regions and in this case as a decision we 
        take the index of boundary region for which the strength of corresponding 
        rule is maximal. In such a case we take into account the rules which are possible.
    \item In other cases: no rule was found or few rules have the same strength
        factor unknown pattern $x$ is rejected.
\end{enumerate}
\subsection{Rough sets algorithm construction with modification of decision rules}
\label{cha:Algorithm_construction_rough_set_modification}
It can happen that for certain number of intervals we cannot find patterns in
the learning set so as a consequence we generate dummy rule, useless in the
classification process. The main drawback of algorithm presented in \label{cha:Algorithm_construction_rough_set}
that is starts with an arbitrary step of granulation and its accuracy strongly
depends on it. In this section the recursive modification of the previous algorithm is
presented allowing for automatically changing the step of granulation is
pattern $x$ is rejected. The modification is as follows:
\begin{enumerate}
    \item Algorithm starts with an arbitrary chosen step of granulation.
        Generally, it is a high value to ensure that recursion can be invoked
        by decreasing $G$ value. Shortly speaking, we divide every domain of feature into $G$
        intervals. The whole procedure 1-6 is repeated from the previous point.
    \item If for the pattern $x$ we cannot find neither certain nor possible
        decision rule, it means that algorithm cannot find proper representation 
        in learning set. Then we try to find matching rule by increasing recursively
        the current interval for every condition attribute $l=1, \ldots, q$ and
        the learning procedure is invoked once again until the proper rule is
        found.  
    \item Recursive algorithm stops if for every attribute $G=1$. Then the
        decision is taken randomly.
    \item To enhance the process of finding proper decision formula for
        different $G$ the decision set $FOR(C)$ for the certain $G$ are stored
        in the memory.
\end{enumerate}

\subsection{Fuzzy logic algorithm construction}
\label{cha:Algorithm_construction_fuzzy_logic}
\subsubsection{Problem formulation}
\label{cha:Fuzzy_logic_basic_problem_formulation}
In the fuzzy logic algorithm one of the most important key is creation of rule
which will ensure proper classification. When we have no expert knowledge about
dataset it is not an easy task to generate them on its own. In the literature
one can find many practical examples of how to generate $IF-THEN$ rules, from
the statistical tools to heuristic algorithms. In this thesis genetic
algorithm will be used to generate rule set as one of the approach in the
recent year (more information about genetic algorithm can be found in section
\ref{cha:Genetic_algorithm}).
For the algorithm construction we have to make few assumptions:
\begin{itemize}
    \item The same as for rough sets algorithm, we assume that we have $N$
        training patterns.
    \item A set $F$ of linguistic values and their membership functions is given
        for describing each attribute. 
\end{itemize}
I think that the second point requires in-depth explanation. First of all how
to partition each attribute into linguistic values and how to describe each
membership function. Here is the place for genetic algorithm to find optimal
parameters. 
\begin{figure}[H]
    \begin{center}
        \includegraphics[width=\textwidth]{fig/fuzzy_example.png}
    \end{center}
    \caption{Example fuzzy partition set for an attribute}
    \label{fig:fuzzy_example}
\end{figure}
Fig. \ref{fig:fuzzy_example} shows an example how to generate fuzzy set $F$ of
possible membership functions. In this example $14$ membership functions can be
used. Each function has a subscript defining its linguistic value. To emphasize
with which problem we facing with let take into account that each attribute is divided 
in the same way as depicted in Fig. \ref{fig:fuzzy_example}. Having
$d$-dimensional feature space possible combination of membership functions for
rule generation is equal to $14^d$. To evaluate each solution is
computationally impossible in a reasonable time. 

In this section we reconsider fuzzy rules in the following form:
$$IF\, x_1=A_{q1}\, AND\, x_2=A_{q2}\, AND\, \ldots\, AND\, x_d=A_{qd}\, THEN\,
class\, C_q\, with \, CF_q$$
where $x=(x_1, x_2, \ldots, x_d)$ is a $d$-dimensional pattern vector; $A_{qi}$
is an antecedent fuzzy set with a linguistic label (taking into account the
example from \ref{fig:fuzzy_example} $A_{qi}$ can be one label from the set
$\{1, 2, \ldots, 9, a, b, \ldots, e \}$); $C_q$ is a consequent class and can
have value one of $\{1, \ldots, m\}$, $CF_q$ is a rule strength determined in the
training phase (see more in section \ref{cha:Fuzzy_logic_rule_generation}); $q$
determines the number of rule from the set $\{1, \ldots, N_{rule}\}$, generally
$N_{rule}$ is about ten or twenty. Rule strength $CF_q$ is used in the
defuzzyfication process.
\subsubsection{Rule generation}
\label{cha:Fuzzy_logic_rule_generation}
The process of rule generation for fuzzy logic without the expert knowledge is
complex and done in couple steps. To construct fuzzy rule we use available
training set. Let say that at first we generate randomly $N_rule$ rules. For
each training pattern $x_p$ calculate the compatibility grade of a single rule
connected with antecedent part $A_q = (A_{q1}, A_{q2}, \ldots, A_{qd})$ using
the product operator of each membership function $\mu_{A_{qi}}$ determined for
$A_{qi}$:
\begin{equation}
    \mu_{A_q}(x_p)=\mu_{A_q1}(x_p)\cdot\mu_{A_q2}(x_p)\cdot\ldots\mu_{A_qd}(x_p)
    \label{eq:mu_product}
\end{equation}
If we know how to calculate the compatibility grade of each training pattern
now we can determine $C_q$ and $CF_q$ for each rule. The fuzzy probability
$P(class\, h|A_q)$ of class $h$, $h=(1, \ldots, m)$ is given by eq. (\ref{eq:fuzzy_probability})
\begin{equation}
    Pr(class \, h|A_q) = \frac{\sum\limits_{x_p \in class\,
    h}\mu_{A_q}(x_p)}{\sum\limits_{p=1}^m\mu_{A_q}(x_p)}
    \label{eq:fuzzy_probability}
\end{equation}
For the rule $R_q$ the label of class is assigned according to eq.
(\ref{eq:fuzzy_max})
\begin{equation}
    R_q: C_q = max\limits_{h=\{1, \ldots, m\}}\{Pr(class\,h|A_q)\}
    \label{eq:fuzzy_max}
\end{equation}
In the learning phase it can happen that that rule $R_q$ can be activated by
patterns coming from different classes. We have to determine the strength
factor for this rule if we have chosen a proper class $h$ for rule $R_q$. 
\begin{equation}
    R_q: CF_q=Pr(class\, h|A_q) - \sum\limits_{h=1, h\neq C_q}^MPr(class\,h|A_q)
    \label{eq:fuzzy_strength}
\end{equation}
If $CF_q$ in eq. (\ref{eq:fuzzy_strength}) is negative then rule $R_q$ is
denoted as dummy and not used in later reasoning.
\subsubsection{Fuzzy reasoning}
\label{cha:Fuzzy_reasoning}
Let assume that $N_{rule}$ fuzzy rules are generated with indicators $C_q$,
$CF_q$ determined by eq. (\ref{eq:fuzzy_max}, \ref{eq:fuzzy_strength}). Then
the process of classification is done as follows:
\begin{equation}
    \Psi(S, x_p) =  C_q \leftarrow max_{h=\{1,\ldots, M\}}\{\mu_{A_{q}}(x_p)\cdot CF_q\}
    \label{eq:fuzzy_classification}
\end{equation}
The label of the class for unknown pattern is determined by a winner rule $R_w$
that has the maximum compatibility grade and the rule strength $CF_q$.

If multiple fuzzy rules have the same maximum product $\mu_{A_q}$ but different
consequent classes then the classification is rejected. The same action is
taken if no fuzzy rule is compatible with the incoming pattern $x_p$. 
\subsubsection{Genetic algorithm for  fuzzy algorithm construction}
\label{cha:Fuzzy_logic_genetic_algorithm}
In this section genetic algorithm will be described in greater details. This
algorithm was used to generate initial number $N_{rule}$ of fuzzy rules for
classification. 
Basic assumptions:
\begin{itemize}
    \item Fuzzy rule encoding is the same as presented in section \ref{cha:Fuzzy_logic_basic_problem_formulation}
    \item Training data set is given
    \item Triangular membership functions are used and described by 2-tuple
        $(a, b)$, where $a$ is the center value, and
        $b$ determines left and right extend of function respectively.
        \begin{equation}
            f(x) = 
            \begin{cases}
                \frac{-1}{b}\cdot x + \frac{a+b}{b} &
                x \geq a \, and \, x \leq (a+b) \\
                \frac{1}{b}\cdot x - \frac{a-b}{b} &
                x \geq (a - b)\, and\, x < a \\
                0 & otherwise
            \end{cases}
            \label{eq:fuzzy_function}
        \end{equation}
    \item Possible partitions of the feature space are determined in the same
        way as in the example presented in fig. \ref{fig:fuzzy_example}
    \item Genetic algorithm uses standard operations such as cross\_over,
        mutation, population generation, fitness evaluation.
    \item As the template for genetic fuzzy algorithm Mitchigan approach was
        used which means that we have $N_{rule}$ number of individuals in the
        population
\end{itemize}
Next few step will present the whole structure of genetic algorithm used in
this section:
\begin{itemize}
    \item Chromosome representation and encoding: 
        \begin{itemize}
            \item Each individual represents a single fuzzy rule $R_q$.
                The length of the chromosome is the same as the number of 
                attributes describing the pattern $x$. Each allele has value
                determining which linguistic variable is used in the current
                rule. Reconsider Iris dataset which is a $4$-dimensional
                classification problem and that for each attribute $14$
                membership functions plus one variable telling to omit the
                attribute (called $DON'T \,USE$). An exemplary individual can be as follows:
                $$1|c|DON'T \;USE|4||1|0.85$$
                Above individual can be decoded into rule presented in fig.
                \ref{fig:fuzzy_rule}
                \begin{figure}[H]
                    \begin{center}
                        \includegraphics[width=\textwidth]{fig/fuzzy_rule.png}
                    \end{center}
                    \caption{Example rule decoded from the individual
                    chromosome (attribute $x_3$ was omitted in this rule)}
                    \label{fig:fuzzy_rule}
                \end{figure}  
        \end{itemize}
    \item Individual evaluation
        \begin{itemize}
            \item To ensure proper genetic algorithm process an appropriate
                fitness function must be defined. Firstly the nature of pattern
                recognition task must be taken into account and secondly the
                structure of the fuzzy algorithm. Generally, we aiming at
                generating rule with the highest $CF_q$ grade, smallest number
                of attributes and the highest classification rate. Fitness
                function is given by eq. (\ref{eq:fuzzy_fitness})
                \begin{equation}
                    F_{fg} = w_1\cdot NC + w_2\cdot NNC + (\frac{1}{NOF})^2 +
                    w_3\cdot CF
                    \label{eq:fuzzy_fitness}
                \end{equation}
                where $w_1$, $w_2$ are weights for a reward and punishment to
                the rule based on the classification result; $NC$, $NNC$ are
                the numbers of correctly recognized and misclassified patterns
                by a particular rule, respectively; $NOF$ is the number of
                attributes used by the rule (in the above example $NOF=3$);
                $CF$ is the strength factor of the rule and $w_3$ is the weight.
                The best individuals are those which maximize function $F_{fg}$
        \end{itemize}
    \item Cross-over, mutation and population generation
        \begin{itemize}
            \item From the whole population two individuals were chosen
                randomly to constitute father and mother parent. With a
                probability of $0.5$ each allele was picked either from mother
                or father chromosome. In this way we generate two new
                individuals.
            \item In the particular generation one chromosome is chosen
                randomly and later in each allele new membership function is
                taken from other possible function. For example is in the first
                allele the first membership function is chosen the set of
                candidates is given by $\{2, \ldots, 9, a, \ldots, e, DON'T\, USE\}$
            \item In genetic algorithm one of the most important issue is how
                to generate next population. Here, after the end of one
                generation individuals from the population are merged with
                those created through cross-over and mutation operations.
                Later, the average fitness value $F_{avg}$ is calculated in the whole
                set. To the next generation those individuals are passed which
                their fitness indicator $F$ is greater than the average $F \ge F_{avg}$
        \end{itemize}
\end{itemize}
Of course it can happen that for cross-over and mutation operator
newly-generated individual will be invalid (the whole chromosome contains only
$DON'T\; USE$ linguistic variables). In such a situation rule is rejected and
the whole generation process is repeated again.

Table \ref{tab:fuzzy_genetic_parameters} presents basic genetic algorithm parameters.
Optimal values were determined during simulations by trial and error method.
\begin{table}[H]
    \caption{Parameter settings for genetic algorithm used in fuzzy logic}
    \centering
    \begin{tabular}{|c|c|}
        \hline
        Parameter & value \\ \hline \hline
        $N_{rule}$ & 10 \\ \hline
        $N_{replace}$ & $N_{rule}/2$ \\ \hline
        Crossover probability & 0.9 \\ \hline
        Mutation probability & 0.3 \\ \hline
        Generations & 500 \\ \hline
    \end{tabular}
    \label{tab:fuzzy_genetic_parameters}
\end{table}
\subsection{Multistage hybrid algorithm construction}
\label{cha:Multistage}
In this section the main algorithm in this thesis will be described. 
\subsubsection{Motivations}
\label{cha:Mutlistage_motivations}
When we deal with complex data it can happen that a single classifier is not
sufficient. There arises a question if connection of different classifier will
improve the classification. In this thesis hybridization of rough sets and
fuzzy logic is presented. The next subsections show algorithm construction.
\subsubsection{Rough sets and genetic algorithm}
\label{cha:Multistage_rough_genetic}
Rough sets algorithm presented in section
\ref{cha:Algorithm_construction_rough_set} uses an arbitrary chosen step of
granulation which affects the accuracy of classification. Additionally, each
attribute uses the same granulation interval which in some cases gives good
results in other efficiency is low. Another drawback of the basic rough set
algorithm is that it uses all attributes for rule construction. 

Finding the optimal attribute reduct and rough set partition is $NP$ problem.
To overcome this obstacle a genetic algorithm was used in the similar way as in section
\ref{cha:Fuzzy_logic_genetic_algorithm}. Now, a single individual describes the
partition for each attribute independently. Reconsider individual encoding for
$4$-dimensional Iris dataset. The number of granulation for the attribute is
chosen from the set $\{1, 2, \ldots, K_{max}\}$, where $K_{max}$ is the maximum
value of discretization. Additionally, $DON'T\; USE$ variable is used to
determine that a given attribute is not used in the rule. The example of
individual is given below:
$$|2|DON'T \; USE|K_{max}|3||120$$
It means that the first feature is divided into two intervals, the second is
not used and the third and fourth are discretized into $K_{max}$ and 3
intervals respectively. The fitness indicator of this individual is $120$.
To evaluate individual the following fitness function given by eq.
(\ref{eq:rough_fitness}) is used.

\begin{equation}
    F_{rg} = w_1\cdot NC + w_2\cdot NNC + (\frac{1}{NOF}) +w_3\cdot (\frac{1}{NOCR})^2
    \label{eq:rough_fitness}
\end{equation}

where $w_1$, $w_2$ are weights for a reward and punishment to
the individual on the classification result; $NC$, $NNC$ are
the numbers of correctly recognized and misclassified patterns; $NOF$ is the number of
attributes used by the rule (in the example above $NOF=3$);
$NOCR$ is the number of certain rule which are derived from
partition given by the individual; $w_3$ is the weight.

The whole procedure can be summarized in few steps:
\begin{enumerate}
    \item Determine maximum partition value for each attribute $K_{max}$. In
        this thesis $K_{max}$ is the same for all features
    \item Generate $N_{pop}$ individuals by randomly assigning value from the
        set $\{1, 2, \ldots, K_{max}, \, DON'T\; USE\}$ to each allele
    \item Treat each individual as a rough set partition and calculate lower,
        upper approximation and boundary region. Evaluate individual using
        fitness function $F_{rg}$
    \item Generate $N_{replace}$ individuals using genetic operators and merge
        with the current population
    \item Choose $N_{pop}$ individual for the next generation
    \item If stopping criteria is not fulfilled go to $2$
\end{enumerate}
Parameters for genetic algorithm used in this section are presented in table
\ref{tab:rough_genetic_parameters}:
\begin{table}[H]
    \caption{Parameter settings for genetic algorithm used in rough sets}
    \centering
    \begin{tabular}{|c|c|}
        \hline
        Parameter & value \\ \hline \hline
        $N_{pop}$ & 10 \\ \hline
        $N_{replace}$ & $N_{pop}/2$ \\ \hline
        Crossover probability & 0.9 \\ \hline
        Mutation probability & 0.2 \\ \hline
        Generations & 100 \\ \hline
    \end{tabular}
    \label{tab:rough_genetic_parameters}
\end{table}
In case of genetic algorithm for rough set $100$ generations were sufficient to
obtain reliable results.
\subsection{Hybrid rough sets and fuzzy logic}
\label{cha:Multistage_rough_fuzzy}
Multistage classifier created in this thesis can be divided into three phases:
\begin{enumerate}
    \item Rough sets classifier construction using genetic algorithm presented
        in section \ref{cha:Algorithm_construction_rough_set}
    \item Fuzzy logic classifier construction with rule generation by heuristic
        approach described in \ref{cha:Fuzzy_logic_genetic_algorithm}
    \item Pattern recognition using hybrid classifier 
\end{enumerate}
Each step plays an important role in the whole process and affect the final
classification accuracy. Proper parameters of genetic algorithm are especially
important. The whole hybrid algorithm can be summarized in the following steps:
\begin{enumerate}
    \item Divide available dataset into three separated subsets: the first for
        genetic algorithm operation, the second and third as training and
        testing.
    \item Train rough sets, fuzzy logic classifiers
    \item Classify pattern using rough sets algorithm:
        \begin{itemize}
            \item If pattern is classified by a certain or possible rule then
                it is a final label
            \item If no rule were found or more than one rule have the same
                strength, but different label then pattern is rejected and
                processed by fuzzy logic classifier.
        \end{itemize}
\end{enumerate}
Illustrative scheme of hybrid classifier is presented in fig.
\ref{fig:schematic}
\begin{figure}[H]
    \begin{center}
        \includegraphics[width=\textwidth]{fig/diagram.png}
    \end{center}
    \caption{Schematic diagram how hybrid classifier works}
    \label{fig:schematic}
\end{figure}


\newpage \clearpage
\addcontentsline{toc}{section}{Literature}
\addcontentsline{toc}{section}{List of figures}
\addcontentsline{toc}{section}{List of tables}


\begin{thebibliography}{99}

\bibitem{bib1}
Cantu-Paz E., Goldberg D. E., : \textit{``Efficient Parallel Genetic Algorithm: Theory
and Practice''}, Department of Computer Science and Illinois Genetic Algorithms Laboratory
University of Illinois at Urbana-Champaign, 1999

\bibitem{bib2}
Gang W., Terrence W. D., Goodman D. E., : \textit{``Optimization of a~GA and
within a~GA for a~2-dimensional layout problem''}, First International Conference on
Evolutionary Computation and its Applications , 1996

\bibitem{bib3}
	Gangming L., Xin Y. , : \textit{``Parallel genetic algorithm on PVM''},
Computational Intelligence Group, Department of Computer Science,
University College, The University of New South Wales, 1996

\bibitem{bib4}
	Cantu-Paz E., : \textit{``A~survey of parallel genetic algorithms''},
Department of Computer Science and Illinois Genetic Algorithms Laboratory
University of Illinois at Urbana-Champaign, 1996

\bibitem{bib5}
	Verma A. , Venkataraman S., Goldberg D. E, : \textit{``Scaling eCGA Model
	Building via Data-Intensive Computing''},
IlliGAL Report No. 2010001, 2010

\bibitem{bib6}
	Xia S. , Jamshidi M., : \textit{``A Genetic Algorithms - Discrete Event Simulation
  Methodology for Modeling and Simulation of Autonomous Systems''},
 Department of Electrical and Computer Engineering and Autonomous Control Engineering
 (ACE) Center, University of New Mexico, Albuquerque, NM 87131, 1999

\bibitem{bib7}
	Moin N. H., : \textit{``Hybrid Genetic Algorithms for Vehicle Routing Problems
	with Time Windows''}, Institute of Mathematical Sciences University of Malaya
       50603 Kuala Lumpur Malaysia, 2000
	   
\bibitem{bib8}
	Vertanen K., : \textit{``Genetic Adventures in Parallel: Towards a~Good Island Model under PVM''}, 
	Department of Computer Science, Oregon State University 303 Dearborn Hall,
	Corvallis OR 97330 USA, 2000

\bibitem{bib9}
	Ogrodowczyk R., Murawski K., Bielecki B., : \textit{``Systemy rozproszone''}, 
	Katedra Informatyki Państwowa Wyższa Szkoła Zawodowa w Chełmie, 2005

\bibitem{bib10}
	Nowostawski M., Poli R., : \textit{``Parallel Genetic Algorithm Taxonomy''}, 
	KES’99, MAY 13, 1999

\bibitem{bib11}
	Łukasik S., : \textit{``Parallel genetic algorithms for graph coloring
	problem using message passing paradigm''}, Journal of Electrical
	Engineering, Vol. 56, No. 12/s, 2005, 1-5

\bibitem{bib12}
	Nazan K., : \textit{``Population Sizing in Genetic and Evolutionary Algorithms''},
	 Illinois Genetic Algorithms Laboratory Department of~General Engineering University 
	 of~Illinois at~Urbana-Champaign, 2003

\bibitem{bib13}
	Geist A., Beguelin A., Dongarra J., Jiang W., : \textit{``PVM: Parallel
	Virtual Machine A~Users' Guide and Tutorial for Networked Parallel
	Computing''}, The MIT Press Cambridge, Massachusetts London, England, 1994

\bibitem{bib14}
	Alba E., Laguna M., Laque G., : \textit{``Workforce Planning with a~Parallel
	Genetic Algorithm''}, Dept. de Lenguajes y Ciencias de la
           Computacion ETS Ingenieria Informática Univ. de Malaga, 1999

\bibitem{bib15}
	Tomassini M., : \textit{``Parallel and Distributed Evolutionary Algorithms:
	A~Review''}, University of Lausanne, Switzerland, 1998

\bibitem{bib16}
	Wang L., Maciejewski A., Siegel H. J., Eldridge B. D., Brad L. Miller :
	\textit{``A~study of five parallel approaches to a~genetic
	algorithm for the Traveling salesman problem''}, Electrical and Computer
	Engineering Department, Computer Science Department, Colorado State University
    Fort Collins, CO 80526-1373, 2000

\bibitem{bib17}
	Harik G., Cantu-Paz E., Goldberg D. E, Miller B. L., : \textit{``The Gambler’s Ruin Problem, Genetic Algorithms, and the Sizing of Populations''}, 
	Illinois Genetic Algorithms Laboratory University of Illinois Urbana, IL
	61801 USA, 1999

\bibitem{bib18}
	Chen J., : \textit{``Theoretical Analysis of Multi-Objective Genetic
Algorithms - Convergence Time, Population Sizing, and Disequilibrium''}, 
Department of Information Engineering and Computer Science Feng Chia University,
Taichung, Taiwan 407, ROC, 2000

\bibitem{bib19}
	Cantu-Paz E., : \textit{``Topologies, Migration Rates, and Multi-Population
	Parallel Genetic Algorithm''}, IlliGAL Report No. 99007 January, 1999

\bibitem{bib20}
	Levine D., : \textit{``User Guide to the PGAPack Parallel Genetic Algorithm
	Library''}, Argonne National Laboratory, 9700 South Cass Avenue Argonne, IL
	60439, 2000

\bibitem{bib21}
	Cantu-Paz E., : \textit{``A Summary of Research Parallel Genetic
	Algorithms''}, IlliGAL Report No. 95007, July 1995

\bibitem{bib22}
	Cantu-Paz E., : \textit{``Designing Efficient and Accurate Parallel Genetic
	Algorithm''}, IlliGAL Report No. 99017, July 1999

\bibitem{bib23}
	Kajan S., Sekaj I., Oravec M., : \textit{``The use of matlab parallel
	computing toolbox for genetic algorithm-based mimo controller design''},
	Institute of Control and Industrial Informatics,
    Faculty of Electrical Engineering and Information Technology, 
	Slovak University of Technology in Bratislava, 2003

\bibitem{bib24}
	Cantu-Paz E., Goldberg E., D., : \textit{``Predicting speedups of idealized
	bounding cases of parallel genetic algorithms''},
	Department of Computer Science and Illinois Genetic Algorithms Laboratory, 
	University of Illinois at Urbana-Champaign, 2001

\bibitem{bib25}
	Smutnicki C., Molga M., : \textit{``Test functions for optimization needs''},
	Wroclaw University of Technology, 2001

\bibitem{bib26}
	Cargal J., M. : \textit{``Gambler's ruin problem''},
	Chapter 33, Discrete Mathematic for Neophytes: Number theory, Probability,
	Algorithms, and other stuff, 1999

\bibitem{bib27}
	Qi J., Burns G. : \textit{``The Application of Parallel Multipopulation
	Genetic Algorithms to Dynamic Job-Shop Scheduling''},
	Department of Engineering, Glasgow Caledonian University, 2003

\end{thebibliography}


\newpage

\listoffigures

\listoftables


\newpage\clearpage
\appendix
\appendix
\makeatletter
\def\Pref@section{Appendix~}
\def\@seccntformat#1{\csname Pref@#1\endcsname \csname the#1\endcsname\quad}
\makeatother

\section{Program description}
\label{Appendix}
\subsection{Installation requirements}
For the test purposes the simulator in \textit{Python} language was written. To
successfully ran this program few requirements must be met. In this master
thesis this software was executed on Linux platform and here this approach will
be described. Of course it is possible to run the program on Windows, but
proper preparation must be undertaken. Requirement for the software:
\begin{itemize}
    \item \textit{Python} in version at least 2.6
    \item NumPy
    \item SciPy
    \item mlpy library with Gsl. Steps for the proper installation:
        \begin{enumerate}
            \item sudo apt-get install python2.6-dev
            \item go to: \textit{http://www.gnu.org/prep/ftp.html}
            \item click on an ftp link close to your location
            \item find the gsl/ directory and click on it
            \item find the gsl-VERSION.tar.gz file, where version is 1.14 or greater. Click on that file to download it.
            \item In a terminal window extract the tar.gz file using tar -xzf gsl-VERSION.tar.gz and then cd to the ./gsl-VERSION 
                directory
            \item Look at the INSTALL file. It will probably tell you to run ./configure, then make, and then make install
            \item download mlpy from \textit{http://sourceforge.net/projects/mlpy/files/}
            \item unzip file and inside directory run from command line python setup.py install
        \end{enumerate}
\end{itemize}

The whole project is divided into modules and each classifier is implemented as
python class:
\begin{itemize}
    \item BasicClassifiers- this class implements basic classifiers such as:
        LDAC, 3-KNN, MaximumLikelyHood Classifier, Gini Index Classifier, svm
        Classifier
    \item RoughSetsClassifier- this class implements basic rough sets
        classifier. Depending on the chosen module it is an algorithm with
        modification of decision rules or not.
    \item GeneticFuzzyLogicClassifier- this class simulate genetic fuzzy logic
        classifier. In the beginning genetic algorithm is run to obtain the
        best decision rule set and later classification is done
    \item GeneticRoughSetsClassifier- this class implements genetic rough sets
        classifier. It comprises of two parts:
        \begin{itemize}
            \item genetic algorithm for obtaining an optimal partition for each
                feature
            \item classification procedure which uses partition from the
                previous step for pattern recognition
        \end{itemize}
    \item HybridClassifier- this class implements hybrid classifier. This is a
        multistage classifier in which rough sets algorithm is treated as the
        first classifier and fuzzy logic as the second.
\end{itemize}
\subsection{Example usage}
To run each classifier few basic steps must be done. First of all proper
parameters with cross-validation and dataset type must be chosen. Below, a
simple example is presented showing how to run genetic rough sets classifier
fir iris dataset
\lstinputlisting[language=Python]{code/example.py}


\end{document}
