\thispagestyle{empty}
\begin{center}
	Abstract
\end{center}
This paper concerns the problem of pattern recognition. Advancement in
computers has spawned the development of artificial intelligence science. In the
recent years many scientists have proposed many algorithms for solving complex
and high dimensional problems. Some solutions give good results, but very
often they are devoted to specific problems. Until now no one has managed to
invent classifier with 100\% classification accuracy. Properties of single
classifiers have been deeply exploited and it is hard to do something more.
Nowadays, it is common to take advantage the diversity of classifiers and build 
hybrid classifier from the basic ones.

This master thesis focuses on two algorithms for pattern recognition: rough
sets and fuzzy logic. Firstly, the basic algorithm properties are presented and
later tests on real datasest are performed to check the classification accuracy
of each classifier. The final result is the recommendation of multistage hybrid
classifier.

All tests were performed on Linux platform and the simulator was written in
\textit{Python} language. As the input to the system real datasets from
\textit{UCI} repository were taken. To ensure diversity in simulations datasets
with different number of attributes were chosen. To divide dataset into
training and testing patterns 4-fold cross validation was used, which means
that in each run data were splitted into 4 even parts and one subset was
treated as a testing and the remaining ones as training and it was done four
times. To ensure more reliable results each algorithm configuration repeated
ten times.

This paper is divided into two parts.  In the first, the review of
literature and the basic notation used in the whole thesis is presented. Section
\ref{cha:Introduction} is the source of the basic
knowledge about pattern recognition. Sections \ref{cha:Rough_set}, \ref{cha:Fuzzy_logic}
present the description of algorithms used in this thesis. The second part,
which is the main point of this thesis, presents experiments analysis. Testing
environment setup and program used in simulations are described in section
\ref{cha:ExperimentAnalysis}, while the results with comments are placed in
section \ref{cha:Simulation_investugations}. Proposal of future work as the 
continuation of this topic and the final summary can be found in
sections \ref{cha:Summary}, \ref{cha:FutureWork}.

Powiedziec jeszcze tutaj o wynikach.


