\thispagestyle{empty}
\selectlanguage{polish}
\begin{center}
	Streszczenie
\end{center}
Niniejsza praca porusza problem klasycznego rozpoznawania obiektów (ang.
pattern recognition). Istnieje wiele metod, a także dziedziń, w których można
zastosować tę wiedzę: od rozpoznawania obrazów, chorób medycznych, czy też
obiektów opisanych za pomocą różnych miar. Rozwój komputerów spowodował 
znaczący rozwój sztucznej inteligencji. Proponowane rozwiązania są bardzo 
zaawansowane i uwzględniają wiele czynników związanych z
określonym problemem. Naukowny prześcigają się w swoich osiągnięciach, przez wiele lat 
dopracowali algorytmy rozpoznawania w najdrobniejszym szczególe jednak jeszcze
nikt nie zdołał stworzyć algorytmu, który nigdy nie popełnia błędu. W ostatnich
czasach znacznie większym zainteresowaniem cieszą się hybrydowe rozwiązania, w
których pojedyncze klasyfikatory zostają połączone w jeden. Głównym założeniem
jest wykorzystanie różnorodnych klasyfikatorów, które popełniają błędy w
różnych obszarach przestrzeni cech, lub też stosują różne wnioskowanie na
podstawie dostępnych danych. W takich sytuacjach istnieje możliwość, że zbiór
klasyfikatorów będzie w stanie się wzajemnie uzupełniać, dzięki czemu wydajność
hybrydowego klasyfikatora będzie znacznie lepsza. W literaturze istnieje wiele
przykładów hybrydowego łączenia klasyfikatorów, natomiast w tej pracy zaproponowano 
wieloetapowy klasyfikator wykorzysujący zbiory przybliżone (ang. Rough Sets) i zbiory 
rozmyte (ang. Fuzzy logic).
 
Praca została podzielona na dwie części. W pierszej z nich zaprezentowany
został przegląd literatury, a także przykłady dostępnych rozwiązań w dziedzinie
rozpoznawania i klasyfikacji obiektów. Rozdział \ref{cha:Introduction} stanowi wprowadzenie do
tej tematyki, szczególnej uwadze należy
poświęcić rozdziały \ref{cha:Rough_set}, \ref{cha:Fuzzy_logic}, w których
dokładnie opisano algorytmy Rough Sets i Fuzzy Logic, ich główne zalety, wady,
a także możliwe zastosowania. W rozdziale \ref{cha:Algorithm_construction}
opisano sposób implementacji algorytmów wykorzystanych w tej pracy
magisterskiej. W drugiej części przedstawiono wyniki przeprowadzonych badań.
Rozdział \ref{cha:ExperimentAnalysis} zawiera informacje o danych testowych, a
także sposobie oceny efektywności algorytmów. Sposób przeprowadzenia badań,
środowisko symulacyjne, a także wyniki testów wraz z krótkim omówieniem zostały 
zamieszczone w rozdziale \ref{cha:Simulation_investugations}. Całość pracy
zakończona jest podsumowaniem, a także propycją przyszłych badań stanowiących
rozwój tego projektu (rozdziały \ref{cha:Summary}, \ref{cha:FutureWork}).

Rough sets, czyli teoria zbiorów przybliżonych została zaproponowana w 1982~r.
przez prof. Zdzisława Pawlaka jako rowinięcie klasycznej teorii zbiorów, w
której element albo należy do określonego zbioru lub nie (jednoznaczna decyzja
true albo false). Zbiór przybliżony opisywany jest za pomocą dwóch zbiorów:
\begin{itemize}
    \item przybliżenie dolne- zbiór obiektów, które na pewno należą do klasy
        decyzyjnej $X$,
    \item przybliżenie górne- zbiór obietków, ktore być może należą do klasy
        decyzyjnej $X$,
\end{itemize}
Zbiory przybliżone zostały utworzone w celu aproksymacji pojęć (zbiorów), ze
względu na niedostateczny stan wiedzy wynikający z braku inforacji o
obiekcie czy też niewystarczających pomiarów. Do głównych zalet zbiorów
przybliżonych w systemach decyjnych można zaliczyć:
\begin{itemize}
    \item redukcję danych, zdolność do selekcji kluczowych atrybutów w procesie
        klasyfikacji,
    \item generowanie reguł decyzyjnych $IF-THEN$,
    \item odkrywanie wzorców z danych,
    \item odkrywanie zależności pomiędzy atrybutami opisującymi dany obiekt,
\end{itemize}
W teori zbiorów rozmytych przyjęło się, że reguły decyzyjne są przedstawione
w postaci tabeli decyzyjnej, w której wiersze odpowiadają obiektom, natomiast
kolumny przedstawiają wartości poszczególnych atrybutów. Spośród wszystkich
atrybutów można wydzielić atrybuty decyzyjne, które mówią o klasie danego
obiektów, a także atrybuty warunkowe opisujące cechy obiektu. 

Fuzzy logic, czyli logika rozmyta została zaproponowana przez Lofti Zadeha,
który zauważył, że pomiędzy stanem 0 (fałsz), a stanem 1 (prawda) istnieje
szereg wartości pośrednich, które można opisać za pomocą stopnia przynależności
danego elementu do zbioru. W tym celu została wprowadzona funkcja
przynależności $\mu$, która przybiera wartości z przedziału $[0,1]$. Istnieje wiele
typów tych funkcji, jednak najczęściej stosowane to funkcja trapezowa,
trójkątna i funkcja Gaussa. Zbiory rozmyte zostały stworzone w celu opisania
zjawisk wieloznacznych i nieprecyzyjnych, używanych w języku naturalnym np.
średni wzrost. Bardzo ważną rzeczą w przypadku logiki rozmytej są zmienne
lingwistyczne, które w języku naturalnym określają pewne wielkości wejściowe
lub wyjściowe. Wnioskowanie odbywa się na podstawie reguł decyzyjnych, które
składają z kilku warunków, a także jednego wniosku. 

Jednym z najważniejszych elementów tej pracy magisterskiej są rezultaty
przeprowadzonych za pomocą własnego oprogramowania utworzonego w języku
\textit{Python}. 
Aby umożliwić porównanie wyników i powtórzalność środowiska symulacyjnego
wykorzystane zostały ogólnie dostępne zbiory testowe z repozytorium \textit{UCI}:
\begin{itemize}
    \item Haberman,
    \item Iris,
    \item Wine, 
    \item Thyroid,
    \item Bupa, 
    \item Pima, 
    \item Wdbc,
\end{itemize}
Zostały wybrane różnorodne zagadnienia, szczególnie pod względem złożoności
problemu (liczby obiektów opisujących obiekt), a także temtatyki, od
diagnostyki cukrzycy, raka piersi, wątroby do rozpoznawani typu wina. Każdy z
testów został powtórzony 10 razy, natomiast aby wydzielić zbiór testujący i
uczący zastosowano technikę 4-fold cross-validation, która polega na tym, że w
każdej turze dane wejściowe podzielone są na cztery części. Jedna z nich
traktowana jest zbiór testujący, natomiast pozostałe jako zbiór uczący. Cały
proces powtarzany jest cztery razy.

Głównym celem tej pracy magisterskiej była zaproponowanie hybrydowego
klasyfikatora, wykorzystującego zbiory rozmyte i przybliżone. W tym celu
został przeprowadzony szereg badań mających początkowo na celu zbadanie
podstawowych właściwości zaimplementowanych algorytmów, a na końcu zbiorcze
podsumowanie efektywności hybrydowego rozwiązania. Na potrzeby badań autor
zaimplementował następujące algorytmy:
\begin{itemize}
    \item Podstawowy algorytm Rough Sets z ustalowny krokiem granulacji
    \item Rekursywny algorytm Rough Sets z modyfikacją reguł decyzyjnych
        wynikających ze zmiany kroku granulacji
    \item Hybrydowy klasyfikator wykorzystujący zbiory rozmyte i przybliżone.
        Był to dwuetapowy klasyfikator, w którym jeżeli w pierwszym kroku nie
        znaleziono reguły pewnej lub możliwej to uruchamiany był drugi etap, a
        w nim klasyfikator oparty na fuzzy logic. Całość wymagała
        zaimplementowania dwóch dodatkowych algorytmów:
        \begin{itemize}
            \item Algorytm Rough Sets połączony z algorytmem genetycznym
                wykorzystanym do znalezienia optymalnego kroku granulacji, a
                także selekcji atrybutów,
            \item Algorytm Fuzzy logic połączony z algorytmem genetycznym
                użytym w celu znalezienia odpowiednich reguł decyzyjnych,
        \end{itemize}
\end{itemize}
