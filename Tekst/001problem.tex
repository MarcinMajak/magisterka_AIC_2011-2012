\section{Problem statement}
Genetic algorithms can be computationally expensive as the result of
complexity within particular application or due to the sheer numbers needed to
reach an acceptable solution. In many problems found in the literature the
expense of a genetic algorithm is due to both these factors \cite{bib5},
\cite{bib7}, \cite{bib11}. The genetic algorithm is a global searching process
based on Darwin's principle of natural selection and evolution, so taking into
account this fact many scientists have noticed that program simulating genetic
algorithms can be implemented in various ways, from
simple sequential program to different method of parallelization. The advent of
multiprocessors has spawned a number of $\mathcal{GA}$ implementations which take advantage
of parallelism available in the algorithm(migration process, subpopulations,
diffusion) \cite{bib8}, \cite{bib9}. In recent years, multi-population genetic algorithms $\mathcal{MGA}$ 
have been recognized as being more effective both in speed and solution quality 
than single-population genetic algorithms $\mathcal{SGA}$. Despite of these advantages,
the behavior and performance of $\mathcal{MGA}$ is still heavily 
affected by a~judicious choice of parameters, such as connection topology, 
migration method, migration interval, migration rate, population number and many
others. The main problem in this topic is to find the best topology, configuration
parameters and balance between the potential speedup and number of slaves.
