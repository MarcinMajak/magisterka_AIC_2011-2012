\thispagestyle{empty}
\begin{center}
	Abstract
\end{center}

This paper concerns the problem of parallel genetics
algorithms and evolutionary programs realization. In the last few decades multiprocessor
computers have spawned development of dispere computation processes, to name
only the few it is cloud computing or mesh grid computers. Design of parallel
algorithms can be carried out in two steps. The first is algorithm's structure
analysis and distinction of processes which can be realised in parallel, with
additional consideration of parallelisation profitability. The second step involves algorithm's
implementation and testing environment setup.

Simulations are focused on efficieny analysis between
parallel genetic algorithm and its sequential counterpart, especially in case
of algorithm's duration and solution accuracy. Genetic algorithm was used
because of its structure and its simplicity to realised some methods in parallel
such as: mutation, reproduction, etc. To manage all the simulations program was
written in C++ simulating sequential and parallel genetic algorithms. For
testing, well known in the literature test functions were used mainly De Jong's
functions and in order to fairly evaluate results efficiency indicators were
proposed in this paper. 

All test were performed on Linux platform with the help of pvm(Parallel Virtual
Machine) framework, which allows virtualization of few processors on one machine
called master. Taking into account the stochastic nature of genetic algorithm
each simulation was repeated $200$ times and the final results represent the
average value. 

This paper is divided into two parts. Sections 2~and~3 are the introduction into
parallel genetic algorithm subject matter. There are listed types of parallel algorithm
and method of parallelisation with mathematical justification. Testing
environment setup and program used in simulations are described in section~4,
while the results of experiments with comments are placed in section~5. 
There were used two types of algorithms: the first is simple sequential while 
the second is parallel genetic algorithm with few subpopulations. The main 
goal of simulations was to find the relationship between number of slave units, 
the interval of migration and number of individuals and their impact on algorithm
efficiency. Proposal of future work as the continuation of this topic and the
final summary can be found in sections 6~and~7.

